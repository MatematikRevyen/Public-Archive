\documentclass[a4paper,11pt]{article}

\usepackage{revy}
\usepackage[utf8]{inputenc}
\usepackage[T1]{fontenc}
\usepackage[danish]{babel}


\usepackage{amsmath}
\usepackage{amsfonts}
\usepackage{amssymb}
\usepackage{amsthm}


\revyname{MatematikRevyen}
\revyyear{2017}
% HUSK AT OPDATERE VERSIONSNUMMER
\version{1.0}
\eta{$2$ minutter $50$ sekunder}
\status{Sandsyligvis færdig}

\title{Beyonce}
\author{Dem der ikke selv texer sketches}

\begin{document}
\maketitle

\begin{roles}
\role{X}[Lise] Instruktør
\role{F}[Mads] Forelæser
\role{S}[Nina]Studerende  
\role{B}[Eigil] Beyonce
\role{D1}[Stig] Danser
\role{D2}[Michael] Danser
\role{D3}[Phillip] Danser
\end{roles}

\begin{props}
\prop{Tweedjakke}
\prop{Forelæsersnoter}
\prop{Attachémappe}
\prop{Fjernbetjening}
\prop{Pult}
\prop{4x mandestilletter}
\end{props}


\begin{sketch}

\scene{Fra venstre ift TeXnikken: D1, B, D3 og D2. Lys op. Eigil beynder at synge.}
\says{B} All the single ladies\\
All the single ladies\\
All the single ladies...
\scene{F kommer ind fra sidetæpperne og pauser sangerne med en fjernbetjening.}
\says{F} Kan I se hvad der foregår her?

\scene{Stilhed/råb fra publikum} 

\says{F} Det er helt tydeligt en stor….. fed….. \act{Gestikulerer mod sangeren} langemand til patriarkatet... Næste gang skal vi kigge på Beyoncés indflydelse på racediskrimination i sydstaterne, så jeg forventer I har fået lyttet GRUNDIGT til Love On Top. Gør det gerne flere gange, selvom jeg ved I har travlt - jeg skal gøre opmærksom på, at det er en del af pensum, så det VIL optræde til eksamen. Tak for i dag.

\scene{S kravler op på scenen (her tændes hendes mikrofin), imens forelæseren pakker sine noter sammen.}

\says{S}[Forsigtigt] Undskyld Erik, jeg var med på den måde, Beyoncé ligesom satte fokus på den diskrepans, hun har oplevet som afroamerikansk feminist, og hvordan det ændrede diskursen i det segmenterede amerikanske retssystem. Jeg undrede mig bare over… Hvis nu hun i stedet antager denne udgangsposition \act{Flytter B's mikrofon op i panden på hende (først herefter slukkes lyden for B's mikrofon). Gestikulerer med hånden hen imod forelæseren.} Får man så stadig det samme resultat?

\says{F} Altså, ja, det gælder faktisk mere generelt end jeg gav udtryk for, men så skal du huske også at ændre hér \act{Sætter Stig klar til magarina} og her \act{flytter B's hånd endnu højere op på hovedet. Tager et stort skridt ud imellem danserne}.

\says{S}[Svarer hurtigt, så det ikke virker som om hun ikke vidste det] Nå, ja, selvfølgelig, selvfølgelig.. Altså jeg spørger fordi vi i kurset Michael Jackson 2, så hvordan sort/hvid transformation kan begrænse en ellers divergerende indflydelse på hoftebevægelser \act{flytter på Michaels hofter.} i streetdance på vestkysten. Og argumentationen hér forløber jo på samme måde: Har de to ting noget med hinanden at gøre?

\says{F} Ja, og det I så i dét kursus er så et specialtilfælde, hvor der er den ekstra betingelse at \act{Flytter på Michaels håndled.} også er tilfældet… Giver det mening? 

\says{S} \act{Nikker} Okay, okay, men betyder dét, at der helt generelt gælder at \act{lægger Eigil ned} så længe at \act{retter på alle de andre (inklusiv at fortsætte magarinna på Stig).}

\says{F} \act{Griner lidt for sig selv} Ja, hvis bare det var så vel, men det er umuligt pga. patologiske modeksempler som Kanye West’s optræden på Roskilde festival ‘09. Man ville faktisk også kunne vise at det medfører det velkendte Gulddreng paradoks!\act{Får Philip til at dappe.}

\says{S} \act{Ved tydeligvis ikke hvad det er} Jaa.. Ja.. det har jeg måske.. læst om.. eller..

\says{F}[Snakker mest med sig selv nu] Men faktisk er det en gammel formodning at \act{bevæger en masse på Michaels håndled.} nemlig Chopin-formodningen. \act{Gnækker for sig selv. Fortsætter grinende/glad} Ja, og hvis du skulle gå hen og bevise det, så kan du godt gøre plads på hylden til en MTV Music Award!

\says{S} Ja.. Puh... Det her kursus om Beyoncé virker godt nok lidt uoverskueligt... Jeg håber, at jeg består.
\says{F}  Ellers skal vi nok finde ud af noget \act{Smækker hende i røven med attachémappen}. 
\scene{Lys ned.}






\end{sketch}


\end{document}