\documentclass[a4paper,11pt]{article}

\usepackage{revy}
\usepackage[utf8]{inputenc}
\usepackage[T1]{fontenc}
\usepackage[danish]{babel}


\usepackage{amsmath}
\usepackage{amsfonts}
\usepackage{amssymb}
\usepackage{amsthm}


\revyname{MatematikRevyen}
\revyyear{2017}
% HUSK AT OPDATERE VERSIONSNUMMER
\version{2.0}
\eta{$2$ minutter $15$ sekunder}
\status{Færdig}

\title{Alarmer Larmer}
\author{Dem der ikke selv TeXer sketches}

\begin{document}
\maketitle

\begin{roles}
\role{X}[Freja] Instruktør
\role{M}[Kristian] Studerende
\end{roles}

\begin{props}
\prop{}
\end{props}


\begin{sketch}


\scene{M sidder og arbejder. Bagtæppet skal stå åbent. Alarmen starter meget blidt og vedkommende reagerer efter en kort stund ved at løfte hovedet, men beslutter sig for at man godt kan arbejde videre. Herefter bliver alarmen højere og personen reagerer igen og rejser sig. Idet personen rejser sig, stopper alarmen. Vedkommende sætter sig ned igen og arbejder videre, hvorefter alarmen begynder stille igen. Vedkommende reagerer og skal til at rejse sig endnu en gang, men alarmen stopper og vedkommende slapper af med et lettende suk, hvorefter alarmen starter igen! Han går ud af scenen og op for at smække døren i venstre side af salen. Alarmen er upåvirket. Han går irriteret over i den anden side af salen, og smækker døren. Alarmen er fortsat upåvirket. Han går hastigt tilbage til scenen, lukker bagtæppet, og alarmen stopper. Han sætter sig lettet ned og læser videre. Efter en lille smule tid, kigger han på sit ur, samler sine ting og går ud af bagtæppet. Dette efterlades åbent, og kort efter begynder alarmen igen.}

\scene{Lys ned.}

\end{sketch}

\end{document}

