\documentclass[a4paper,11pt]{article}

\usepackage{revy}
\usepackage[utf8]{inputenc}
\usepackage[T1]{fontenc}
\usepackage[danish]{babel}


\usepackage{amsmath}
\usepackage{amsfonts}
\usepackage{amssymb}
\usepackage{amsthm}


\revyname{MatematikRevyen}
\revyyear{2017}
% HUSK AT OPDATERE VERSIONSNUMMER
\version{1.0}
\eta{$2$ minutter}
\status{Sandsyligvis færdig}

\title{Studiecafésfyren}
\author{Dem der ikke selv texer sketches}

\begin{document}
\maketitle

\begin{roles}
\role{X}[Freja] Instruktør
\role{F}[Christoffer] Fyr
\role{P}[Frederikke] Pige
\role{U}[Mads] Ucharmerende dreng
\end{roles}

\begin{props}
\prop{}
\end{props}


\begin{sketch}

\scene{Lyset går op og en fyr i studiecafé-t-shirt går over til en køn pige. Han sætter sig omvendt på en stol og læner sig flirtende indover. Alle køn er muligvis byttet.}

\says{F} Hvad kan jeg hjælpe dig med?

\says{P} Jo, det er den her opgave, hvor jeg skal angive samtlige cyklis… \act{Afbrydes af F}

\says{F}  Jeg bliver altså nødt til at sige, at du har altså bare nogle af de flotteste øjne, jeg nogensinde har set. Det er som om man aldrig bliver færdig med at se ind i dem. Lidt ligesom himlen, der fortsætter i det uendelige. 

\says{P} Øhh, tak. Det var ehhh, sødt sagt, men \act{Snupper sin pen og skriver} det er mere fordi vi har Z modulo 5 Z og vi skal… \act{Afbrydes af F}.

\says{F}\act{Lægger sin hånd på P's} Du har virkelig en flot håndskrift…. 

\says{P}[Hylet ud af den] Øøøhhhh, Ok, tak, men je… \act{afbrydes af F}

\says{F} Du er simpelthen så smuk, at jeg ikke ved, hvad jeg skal gøre af mig selv. Det er som om jeg glemmer alt om, hvor jeg er og hvad jeg laver..

\says{P} Fint!, men \act{Afbrydes endnu en gang af F}

\says{F} Jeg må simpelthen lære dig bedre at kende! Er du ikke frisk på at tage en drink her i aften, bare os to?

\says{P} Men, jeg ville jo bare vide, hvordan den her opgave om cykliske undergrupper skal løses.

\says{F}[Kommer til sig selv] Nåh, jo. Du skal bare benytte Lagrange 

\scene{U sidder med hånden oppe og F går stille over til den nye, der skal have hjælp. Han snupper en stol, vender den rundt og sætter sig omvendt på den og tager blidt fat i drengens hånd og siger smilende}

\says{F} Hvad kan jeg så hjælpe dig med?

\scene{Lys ned}


\end{sketch}

\end{document}

