\documentclass[a4paper,11pt]{article}

\usepackage{revy}
\usepackage[utf8]{inputenc}
\usepackage[T1]{fontenc}
\usepackage[danish]{babel}

\usepackage{amsfonts}

\revyname{MatematikRevyen}
\revyyear{2017}
% HUSK AT OPDATERE VERSIONSNUMMER
\version{1.0}
\eta{$2.16$ minutter}
\status{færdig}

\title{Lars H.}
\author{Ulrik, Erik den Gamle}

\begin{document}
\maketitle

\begin{roles}
\role{X}[Lasse] Instruktør
\role{LH}[Marius] Lars “can’t Hessel the Holt” Hesselholt
\role{A}[Janne] Lars’ assistent (evt. Nathalie)
\role{HKI}[Christoffer] En tilfældig Halvdårligt Karikeret Institutleder

\end{roles}

\begin{props}
\prop{En lyserød LinAlg bog}
\end{props}


\begin{sketch}

\scene{Lys op.}

\says{HKI} \act{kommer ind på kontoret} Nå, Lars, det gik da strålende!
Over al forventning! Jeg havde jo godt nok været lidt bekymret for, om det var den rigtige opgave til dig, men…
\says{LH} Tænker du på konferencen om “Stable homotopy theory”. Jo, det gik da fint…
\says{HKI} Nej, Lars, jeg tænker på LinAlg-kurset. Beståelsesprocenten var jo højere end nogensinde… og karakterne særdeles pæne!
\says{LH} Hvad? Var der nogen, der bestod?! Blev de slet ikke forvirrede over skævlegemer og polynomiumsringe?
\says{HKI} Jeg tror ikke, at nogen rent faktisk læste de definitioner. Hvis man bare
laver det over $\mathbb{R}$, er lineær algebra jo ret intuitivt.
\says{LH} Shhy... nogen kunne høre dig!
\says{HKI} Men du må da indrømme, at LinAlg-kurset sagtens kan bestås, hvis man bare kan løse lineære ligningssystemer.
\says{LH} Men det var jo netop det, jeg prøvede at skjule.
\scene{HKI klapper Lars på ryggen for at trøste ham}
\says{HKI} Ja, sådan kan det jo gå Lars...  Jeg har forresten også langt mindre artige forelæsere at udskælde.
For eksempel skal jeg råbe lidt af Ernst Hansen over... tja, alle hans kurser.

\scene{HKI forlader scenen. LH sætter sig fortvivlet ned ved sit skrivebord og drikker kaffe. }

\scene{A kommer ind, og bladrer i LinAlg bogen}

\says{A} Lars, Lars. Jeg har tænkt over det vi talte om sidst. Hvis vi nu definerer
determinanten som den entydigt bestemte n-multilineære alternerende form på ringen af
n kryds n-matricer, der sender identitetsmatricen til 1, og så overlader det til
dem selv at indse, at det faktisk er ret nemt at regne den ud.
\says{LH} Nathalie, det er ligegyldigt, alt er tabt. Russerne er bestået.
\says{A} \act{græder}
\says{LH} Nej, der må vildere midler til! \act{Rejser sig og laver vilde fagter med armene}
\says{A} Vildere end en bog, der er lyserød? Hvilke andre farver findes der?

\scene{LH stopper op og kigger på A}

\says{LH} Hvad er der galt med lyserød?!
\says{A} \act{Chokeret} Var… var det ikke et af dine tiltag?
\says{LH} \act{Fornærmet} Det er tilfældigvis bare en mere spændende farve til en bog.
\says{A} Men næste år har vi vel en chance til? Ikk?
\says{LH} Jo, men hvad vi kan gøre anderledes, for at gøre det mere uforståeligt?

\scene{Begge står og tænker sig om. LH begynder at vandre frem og tilbage på scenen.}

\says{A} Kan vi… introducere… nej, det har vi allerede gjort.
\says{LH} Nej… det kan vi nemlig ikke. Men vi kunne jo introducere dem for topologiske vektorrum.
Det ville de med garanti være dårlige til.
\says{A} Jo, men de er jo allerede dårlige til generelle abstrakte vektorrum, og dem indfører vi jo fint!
\says{LH} Vent! Det er jo genialt! Vi kan bare eksaminere i dem!
\says{A} Jo, men hele eksamen kan jo ikke handle om en enkelt del af den sidste halvdel af bogen.
\says{LH} Nej, Nathalie. Kan du ikke se det? Vi har brugt alt denne tid på at eksaminere i de konkrete,
nemme at udregne dele af bogen. Hvis vi skal dumpe russerne, så skal vi bare eksaminere i alt det abstrakte i stedet for!
\says{A} Men Lars, der er altså også fysikere, der tager kurset. Det vil jo slå dem ihjel.
\says{LH} Så meget desto bedre.

\scene{Lys ned.}

\end{sketch}
\end{document}
