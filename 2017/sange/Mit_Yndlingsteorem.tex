\documentclass[a4paper,11pt]{article}

\usepackage{revy}

\usepackage{amsmath}
\usepackage{amsfonts}
\usepackage{amssymb}
\usepackage{amsthm}
\usepackage[utf8]{inputenc}
\usepackage[T1]{fontenc}
\usepackage[danish]{babel}


\revyname{MatematikRevyen}
\revyyear{2017}
\version{0.1}
\eta{$4.26$ minutter}
\status{Færdig}

\title{Mit Yndlingsteorem}
\author{Eigil, Erik, Stolberg}
\melody{Klumben \& Raske Penge: ``Faxe Kondi''}

\begin{document}
\maketitle

\begin{roles}
\role{X}[Freja] Instruktør
\role{X}[Lasse] Instruktør
\role{Y}[Nanna] Koreograf
\role{R1}[Nadia] Rapper
\role{R2}[Line] Rapper
\role{D1}[Josefine] danser
\role{D2}[AK] danser
\role{D3}[Aurora] danser
\role{D4}[Erica] danser
\end{roles}

\begin{song}
\sings{R1+R2}[omkvæd] For det' mit yndlingsteorem
Jeg lover det så nemt at brug', at det' ekstrem'
jeg bruger det hver dag
og jeg kan ik' la' vær'
Selv om min instruktor si'r jeg ik' kan brug' det her

For det' mit yndlingsteorem
Jeg lover det så nemt at brug', at det' ekstrem'
jeg bruger det hver dag
og jeg kan ik' la' vær'
Selv om min instruktor si'r jeg ik' kan brug' det her

\sings{R1} Det har været kendt siden det syttende århundrede
Jeg si'r skud ud til Girard der i sin tid har fundet på det

Giver du bevis ska' du være rigtig grundig
ikke sidde på din røv og drikke faxe kondi
Bli'r du i tvivl, ja så ska' du bare tjek' min homie Gauss
viste det på fire måder, ja,  for han var ornlig Bauss

Pia K. kalder mig for fundamentalist
Og alle dem fra DF, de vil ha' mig udvist
Her i Danmark der skal du nemlig være integrer't.
Jeg siger fuck dem, det MI får jeg aldrig lært
Det har jeg ik' tid til, for mit hoved tænker kun
på at finde $n$ rødder til mit polynomium

\sings{R2} Bachelorstuderende i KomAn bruger Cauchy
Integrer en rationel funktion, så bru'r du Cauchy
Op' på fjerde sidder kloge folk og bruger Cauchy
Ja selv dem der kun ka' li' heltal bruger Cauchy

Jeg ka' godt li' algebra og jeg bestod da' os' MI
Men der' kun et teorem som ka' gøre mig helt fri
Kurveintegralet af en meromorf funktion
Det ka' vi finde nemt det er min konklusion
For du ska' bar' find' poler og så der's residuer
Og så rødder, det' det eneste der dur

Ham Augustin der han var helt vildt klog
Cours d'Analyse det er verdens bedst' bog
Så det' ren KomAn lige siden første blok
MI og FunkAn det er bar' ikk' nok
Lig'meget hva' emnet er i mit projekt
Vær sikker på Cauchy bli'r brugt korrekt

\sings{R1+R2}[omkvæd] For' det' mit yndlingsteorem\dots

\sings{R1} Det var i det gamle Kina, jeg ska' sig' dig at de var gal'.
De fandt frem til en sætning, og den er ikke helt banal.
I et system af kongruenser, der' en vis entydighed.
De hele tal kan demonstrere en vis lydighed.
Jeg syn's ham, der fandt på det, han er verdens største king
Alle elsker ham, han er en helt i Beijing
Hans tanker, de var smart', ja tro mig det var de sgu
Ja det var rentdyrket, talteoretisk kung fu.
Det kom til ham, imens han sad og spiste sine ris
Han så at det var sandt, han mangled' bare et bevis.
Der gik en del år, før der var nogen, der tog sig tid.
Det kræver tankekræfter og en hulens masse flid.

\sings{R2} Hør nu her, selv hvis vi holder os til talteori'n,
så kender jeg en sætning der er sejere end din.
For Fermat var den snarere den sidste end den første.
For mig har den sætning altid vær't den største.
Det' den med de der eksponenter der' større end to
Og alle og enhver de ved den sætning den er go'
Lig' siden jeg var rus der har jeg haft et Fermat-sus
Hvis du disser Andrew Wiles vil jeg begrave dig i grus

\sings{R1} Når jeg skal integrer',
så er det aldrig svært
Jeg husker bare at huske alt det som jeg har lært.
Der' ikke nog't suspens.
Jeg ryger fed i mens
jeg kaster rundt omkring med domineret konvergens.
Hvis du vil være sindssyg
til at integrere med hensyn til så'n et my,
så skal du skynde dig hen og læse MI,
For DCT det er sætning 11.2 i
Schilling.

\sings{R1+R2} For' det' mit yndlingsteorem\dots
\end{song}

\end{document}
