\documentclass[a4paper,11pt]{article}

\usepackage{revy}
\usepackage[utf8]{inputenc}
\usepackage[T1]{fontenc}
\usepackage[danish]{babel}


\revyname{MatematikRevyen}
\revyyear{2017}
% HUSK AT OPDATERE VERSIONSNUMMER
\version{1}
\eta{3 minutter og 54 sekunder}
\status{Færdig}

\title{De Fede Teoremer}
\author{Ulrik '14}

\begin{document}
\maketitle

\begin{roles}
\role{X}[Lise] Instruktør
\role{Y}[Tina] Koreograf
\role{A}[Peter] Matematiker
\role{B}[Stine T] Stakkels, stakkels pige
\role{D1}[Anne] Danser1
\role{D2}[Rikke] Danser2
\role{D3}[Sofie] Danser3
\role{D4}[AK] Danser4
\role{N1}[MaWeK] Ninja
\role{N2}[Stig] Ninja
\end{roles}

\begin{props}
\prop{Rekvisit}[Person, der skaffer]
\end{props}


\begin{song}
\sings{A} På klub kan man ikke høre, hvad jeg siger
   Så på bar, der farer jeg!
   Mig og nog'n folk fra studiet drikker øl
   Det' for al', kom med på leg
   Jeg spotter en pige, der er med på den værste,
   Det værste, af det jeg har lært
   Så tag min hånd, skat, skal prøv' at smelte dit hjerte
   Og tro mig, det bli'r svært, for jeg si'r

\sings{A} Pig', du ved jo nok om pi,
   Men kender du til, algebra og talteori
   Prøv du bar' at slip' mig fri
   Vi kan C stjerner på himlens sti
\sings{B} Hør, du, ta'r nok fejl af mig
   Jeg lukker ned, bar’ du sender plusstykker min vej
   Nørderi, tænder mig ej
   Nørderi, tænder mig ej 
   Nah, nah, nah

\sings{A} Du' forelsket pr. induktion
   Det' ren logik, der' jo ikke nog'n
   Der' ikke skælver ved kridtets ton'
   Af de fed' teoremer!
   For når du kloner din missekat
   Eller du helst vil kvant' i nat
   Aner du, at lykke findes i den skat
   Der' de fed' teoremer!

\sings{B} Hvad når de er helt ubruglig?
\sings{A} Det' de fed' teoremer!
\sings{B} Hvad hvis jeg er helt udulig?
\sings{A} Brug' de fed teoremer!
\sings{B} Hvis forståelsen er gruelig?
\sings{A} Det' for fed' teoremer!
\sings{A} Jeg ved blot, at der er intet stort forfald
I de fed' teoremers kald!

\sings{A} En ug' sener', er hun ikke alene,
   For jeg ta'r hend' da med på date,
   Hun er helt forhippet, så hun virker helt sippet   
   Jeg skriver op, og hun si'r hun er "late"
   Taler i masser af timer, indtil hun nærmest besvimer
   Af ren og skær og ægte fascination
   Da hun skal ha' ambulancen, så griber jeg chancen
   Taler vider' på min telefon, og jeg si'r

\sings{A} Pi', kender du til mål?
   Fourierrækker modellerer din skål
   Din' funktionaler kan tål
   At se en forskel på hvert punkt
\sings{B} Hør du, er da lidt indbilsk
   Jeg elsker tøj, fede biler og actionfilm
   Nørderi, tænder mig ej
   Nørderi, tænder mig ej 
   Nah, nah, nah
   
\sings{A} Du' forelsket pr. induktion
   Det' ren logik, der' jo ikke nog'n
   Der' ikke skælver ved kridtets ton'
   Af de fed' teoremer!
   For når du kloner din missekat
   Eller du helst vil kvant' i nat
   Aner du, at lykke findes i den skat
   Der' de fed' teoremer!

\sings{B} Hvad når de er helt ubruglig?
\sings{A} Det' de fed' teoremer!
\sings{B} Hvad hvis jeg er helt udulig?
\sings{A} Brug' de fed teoremer!
\sings{B} Hvis forståelsen er gruelig?
\sings{A} Det' for fed' teoremer!
\sings{A} Jeg ved blot, at der er intet stort forfald
I de fed' teoremers kald!

\sings{A} Kom nu, kig på formler med mig
Kom nu, kig på formler med mig
Kom nu, kig på formler med mig
Kom nu, kig på formler med mig
Kom nu, kig på formler med mig
Kom nu, kig på formler med mig
Kom nu, kig på formler med mig
Kom nu, kig på formler med mig

\sings{A} Du' forelsket pr. induktion
   Det' ren logik, der' jo ikke nog'n
   Der' ikke skælver ved kridtets ton'
   Af de fed' teoremer!
   For når du kloner din missekat
   Eller du helst vil kvant' i nat
   Aner du, at lykke findes i den skat
   Der' de fed' teoremer!

\sings{B} Vil ik' kig' på formler med dig
Dit nørderi, det' kun for dig
\sings{A} Det de fed' teoremer!
\sings{B} Vil ik' kig' på formler med dig
Dit nørderi, det' kun for dig
\sings{A} Det de fed' teoremer!
\sings{B} Vil ik' kig' på formler med dig
Dit nørderi, det' kun for dig
\sings{A} Det de fed' teoremer!
Jeg ved blot, at der er intet stort forfald
I de fed' teoremers kald!

\end{song}
\end{document}