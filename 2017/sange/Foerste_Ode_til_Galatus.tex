\documentclass[a4paper,11pt]{article}

\usepackage{revy}
\usepackage[utf8]{inputenc}
\usepackage[T1]{fontenc}
\usepackage[danish]{babel}


\revyname{MatematikRevy}
\revyyear{2017}
\version{1.0}
\eta{ca. 2 minutter 30 sekunder}
\status{Færdig}

\title{Første Ode til Galatius}
\author{Josefine '14, Julie '15, Jakob '11}
\melody{Katinka Katinka}

\begin{document}
\maketitle

\begin{roles}
\role{X}[Freja] Instruktør
\role{S}[Tomas] Solist


\end{roles}

\begin{song}
\scene{Lys op. Følgende sang passer med vilje ikke på versefødderne og skal synges derefter.}

\sings{S} I skal høre efter når i har algtop
eksamen er sat til imorgen
Et forår har kysset Galatii krop,
det nytter ik’ at være doven
Slå måsen i sædet, gør klar til et overflødighedshorn
Hvis ik’ du vil dumpe så lyt til mine guldkorn 
nu kommer en liste i alle skal ku
hvis i vil bestå det her kursus


\sings{S} Det først ord er overlejringsafbildning
Og hvad det betyder ved ingen 
Det er noget med en fjeder som der bli’r mast 
men mere end det, det ved ingen.
Men heldigvis er der så mange andre ting
Som for eksempel barycentrisk underdeling
og deformationsretrakt og retrakt og
et kædekompleks og mit simplex! 

\sings{S}[Råbt entusiastisk] Wuhu!!!


\sings{S} Og homotopiækvivalens og Stig
og løkke og fundamentalgruppe
og konkatenering er multilineær
og covering map og Diedergruppen
og quasi- og para- og semi kompakt, 
og kohomologi og homologi og
simpliciel homologi og cellulær homologi
singulær og U-lille homologi

\sings{S}[Sagt entusiastisk] Kom så! Syng med! I skal kunne det imorgen!

\sings{S} Så er der det semilokale enkeltsammenhængende semi-direkte produkt 
af mængde teoretiske grunde.
Og bli’r du i tvivl så frygt ej tensorér!
Og husk $S^n$ er kontraktibel.
Så tæver vi censor til han siger 12
men ikke før du har snakket om
mangfol….dighed
la la la la la la la la la la la
Så nu er i klar til at dumpe %eksamen


\sings{S}[Sagt] Imorgen! I små møgsvin!


\scene{Lys ned}



\end{song}

\end{document}