\documentclass[a4paper,11pt]{article}

\usepackage{revy}
\usepackage[utf8]{inputenc}
\usepackage[T1]{fontenc}
\usepackage[danish]{babel}


\revyname{MatematikRevy}
\revyyear{2017}
\version{1.0}
\eta{2 minutter 54 sekunder}
\status{Færdig}

\title{Hausdorff Eller Ej}
\author{Josefine '14, Erik '15, Jakob '11}
\melody{Tommy Seebach - Krøller Eller Ej}

\begin{document}
\maketitle

\begin{roles}
\role{X}[Lise] Instruktør
\role{Y}[Julie] Koreograf
\role{S1}[Tomas] Solist
\role{S2}[Lotte] Solist
\role{D1}[Nicolas] Danser
\role{D2}[Phillip] Danser
\role{D3}[Rikke] Danser
\role{D4}[Nina] Danser
\end{roles}

\begin{song}
\scene{Lys op. }

\sings{S1} Jeg har et rum, det stort, 
med mange åbne mængder.
\sings{S2} Jeg har et rum, det flot
Og meget algebraisk
\sings{S1} Jamen mit er bare/meget finer’ end dit er
\sings{S2} Okay, måske, men hvem gider ting der er nemt
Når man kan lave geometri 
Med sin topologi

\sings{S1+S2} Hausdorff eller ej – 
vi elsker vores rum
Stigforbundet eller ej – 
vi elsker jeres rum
\sings{S2} Hvad end der er givet -
 Om det har en metrik, er tælleligt,
Og sammenhængende
\sings{S1+S2} Om de' store eller små, 
er Hausdorff eller ej.

\sings{S1} Hvad end der er givet
Om det har en metrik, er tælleligt,
Og sammenhængende
 Om de grove eller fin', 
er Hausdorff eller ej

\sings{S1+S2} Længe leve alle rum
Åbne mængder eller ej – 
vi elsker vores rum
Quasikompakt eller ej – 
vi elsker jeres rum
\sings{S2} Hvad end der er givet -
Om det har en metrik, er tælleligt,
Og sammenhængende
\sings{S1+S2} Om de' store eller små, 
er Hausdorff eller ej.

\sings{S1} Længe leve rum
\sings{S1+S2} Om de er Hausdorff eller ej




\scene{Lys ned}



\end{song}

\end{document}