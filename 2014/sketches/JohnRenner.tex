\documentclass[a4paper,11pt]{article}

\usepackage{revy}
\usepackage[utf8]{inputenc}
\usepackage[T1]{fontenc}
\usepackage[danish]{babel}

\revyname{MatematikRevy}
\revyyear{2014}
\version{1.0}
\eta{$3$ minutter}
\status{Udkast færdigt}

\title{John Renner}
\author{Jakob og Shake}

\begin{document}
\maketitle

\begin{roles}
\role{I}[Jasper] Instruktør
\role{R}[Brandt] Rus
\role{S}[Kristian] Ældre studerende
\role{J}[Malthe] Renner
\role{E} Ernst
\role{D}[Alexander] Datalog
\end{roles}

\begin{props}
\prop{Kommer senere...}[]
\end{props}
  
\begin{sketch}
\scene I venstre side af scenen sidder to studerende, i højreside ses John Renner.


\says{R} Jeg har hørt noget om at vi skal fusioneres med DIKU. Og der er mange der er sure over det, men jeg forstår ikke rigtigt hvorfor.

\says{S} Jo Rus, selve fusionen er dikteret oppefra af Dekan John Renner.

\scene Spot på John, der mimer med.
\scene Episk baggrundsmusik

\says{S} John Renner har fra sit kontor, skuet ud over universitetsparkens vidder. John så to institutter, der gik imod hans vilje, så John sagde:

\says{J} De skal være ét institut.

\says{S} Så John bestilte en rapport over alle fordelene ved et samlet institut. John så, at fusionen var god, og gjorde de to institutter til ét. 

\says{R} Så han har bestilt en god rapport og handlet efter den. Hvad er problemet så?

\says{S} Lad mig stille dig nogle hypotetiske spørgsmål.
Vil du gerne være sikret en siddeplads i S-toget?

\says{R} Øøh, ja da.

\says{S} Vil du gerne kunne komme til tandlæge om lørdagen?

\says{R} Ja, det ville da være meget rart.

\says{S} Vil du gerne have lavere renter i banken?

\says{R} Ja selvfølgelig.

\says{S} Jeg har en løsning på dine problemer. Vi skal simpelthen bare slå alle jøder ihjel.

\says{R} Nejnejnej, det kan vi da ikke.

\says{J} Men ifølge rapporten vil det bringe en masse fordele med sig.

\says{R} Okay, okay, jeg forstår hvad du mener. Man bliver nødt til også at se på ulemperne. Men sådan en fusion har været på tale før, og der var en masse modargumenter. Hvorfor dur de ikke længere?

\says{S} Jo, nu skal du høre, hvad der skete.

\scene Episk baggrundsmusik

\says{S} John læste rapporten, og John så, at fusionen var god. Så John sagde til institutterne:

\says{J} Jeg har allerede hørt og ignoreret jeres modargumenter, så jeg behøver ikke at høre dem igen.

\says{R} Jamen, du skal jo også se det fra John Renners side. Når der er så mange institutter, er der også flere kontaktpersoner. Og det er jo hårdt, når der er så mange mennesker, der alle fortæller ham, at hans ideer er dårlige.
% Der er jo også stor mangel på lokaler. Så giver det da meget god mening, at vi kan dele dem.
% Overgang?

\scene Episk baggrundsmusik

\says{S} John så på datalogerne, og John så, at ingen datalog kunne lave et ordentligt induktionsbevis. Så John besudlede matematikkens fantastiske væsen med datalogernes computeri og gjorde datalogerne til matematikere.

\says{R} Jamen har vi ikke meget tilfælles med datalogerne?

\says{S} Vi kunne også bare lægge fysik og kemi sammen. Kemi er meget vådt, fysik er meget tørt, så tilsammen bliver det jo helt perfekt... Men nej, nej, Dekanen og Prodekanens fag; det rører man ikke ved!

\says{J} Mit fag rører man ikke ved.

\says{R} Er det ikke bare fordi, de ved nok om faget til at kunne se, at det ville være en dårlig ide?

\scene S og J laver fagter i takt.

\says{S} Det er en konspiration siger jeg dig! Uh uh, se mig jeg hedder John Renner, jeg kan finde ud af at fusionere institutter. Uh, uh jeg er så sej. Jeg er over gud!

\says{J} Jeg er over Ernst Hansen!

\scene Stiller sig på billedet.

\says{R} Jeg kender altså flere folk der har mødt John Renner, og han virker altså rar, når man bare har ham på tomands hånd.

\says{S} Ja, det gør Sys Bjerre efter sigende også!

\scene Lys på John, hvor der er kommet to fagfascismeflag op. J rejser sig op og sætter hånden på hjertet.

\says{J} (Synger) Mat'matik approksimeres. Dataloger ignoreres. Institutter fusioneres. Ren fysik det bli'r hvort svar.

\scene Ernst Hansen kommer på scenen og laver Jedi Mind Trick.

\says{E} Der er ikke brug for nogen fusion.
\end{sketch}

\end{document}
