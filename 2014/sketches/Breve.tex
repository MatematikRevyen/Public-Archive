\documentclass[a4paper,11pt]{article}

\usepackage{revy}
\usepackage[utf8]{inputenc}
\usepackage[T1]{fontenc}
\usepackage[danish]{babel}

\revyname{MatematikRevy}
\revyyear{2014}
\version{1.0}
\eta{$2$ minutter}
\status{Udkast færdigt}

\title{Breve}
\author{Jakob og Shake}

\begin{document}
\maketitle

\begin{roles}
\role{I}[Shake] Instruktør
\role{F1}[Brandt] Fan af Eilers
\role{F2}[Marcus] Fan af Jan Phillip Solovej
\role{F3}[Jeppe] Fan af Ernst
\role{A1}[Agent A.] Agent A
\role{A2}[Lasse] Agent B
\end{roles}

\begin{props}
\prop{Breve}[]
\prop{Eilers' sko}[]
\prop{Kommer senere...}[]
\end{props}

\begin{sketch}
\scene{Lys på scenen. F1 Kommer ind på scenen med et brev i hånden og en sko skjult bag ryggen.}

\says{F1} Hej alle sammen, jeg har et brev jeg gerne vil dele med jer. For der er ikke blevet svaret på det.
Kære Søren Eilers

\scene{Spot på Eilers.}

\says{F1 (fortsat)} Sidste år, tog jeg kurset Dis2. Til at starte med kom jeg til forelæsninger for at høre om de interessante grafer. Men som blokken og ugerne gik var der kun en relation der interessede mig. Mit fokus gled stille og roligt fra tavlen, ned mod dine velformede biceps, der elegant svang kridtet. 
I kridtskyen var det som om at dit hår dansede og mit blik gled ned imod dine mægtige størrelse 42. Skoenes sorte lak mindede mig om en mørk nat hvor man varmes af den man holder kær.
Jeg blev nærmest grebet af, jeg ved ikke hvad, og pludselig voksede min egen kridtstang til umenneskelige proportioner.

\scene{F1 bliver afbrudt af R1 og R2 der kommer ind og trækker F1 ud.}

\says{F1 (fortsat)} Min kærlighed er større end noget polititilhold!

\scene{F2 kommer ind.}

\says{F2} Jeg vil gerne, på revyens vegne, have lov til at sige undskyld for det lidt upassende udbrud tidligere, og i den anledning vil jeg gerne levere din sko tilbage Søren

\scene{Går ud til Søren og leverer skoen tilbage.}

\says{F2 (fortsat)} Jeg har, i undskyldningens ånd. skrevet et lille brev:
Kære Jan Phillip Solovej

\scene{Spot på Jan Phillip Solovej.}

\says{F2 (fortsat)} Sidste år var jeg den eneste der mødte op til alle dine forelæsninger i avancerede partielle differentialligner. Jeg forstod selvfølelig ikke et ord af hvad du sagde men du var bare så smuk. 
Åååååh Jan Philip Solovej, nu er jeg så glad for at jeg dumpede. For så får jeg lov til at tage kurset en gang til. Nu har jeg så syv uger mere hvor jeg kan stirre intenst på dig og bare mærke 
(Mand: hvordan min lineære operator divergerer.)
(Kvinde: hvordan mine bukser bliver lige så våde som tavlesvampen.)

\scene{Bliver afbrudt af R1 og R2 og bæres ud.}

\says{F2 (fortsat)} Denne sang den er til dig! ÅÅÅÅÅÅH Jan Philip Solovej.

\scene{F3 kommer ind.}

\says{F3} Jeg vil gerne have lov til at bringe et samlet undskyld på revyens vegne for de to tidligere udbrud der har været i løbet af aftenen. Det er selvfølgelig ikke iorden og Søren (og Jan Philip) du skal selvfølgelig nok få din værdighed tilbage. 
Jeg har derfor som undskyldning skrevet et lille brev.
Kære Ernst Hansen

\scene{Spot på Ernst. R1 og R2 kommer på scenen for at stoppe ham.}

\says{F3} Du er for nice.

\scene{R1 og R2 nikker og går ud. Lys ned.}
\end{sketch}

\end{document}
