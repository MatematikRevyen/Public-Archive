\documentclass[a4paper,11pt]{article}

\usepackage{revy}
\usepackage[utf8]{inputenc}
\usepackage[T1]{fontenc}
\usepackage[danish]{babel}

\revyname{MatematikRevy}
\revyyear{2014}
\version{1.0}
\eta{$1.5$ minutter}
\status{Udkast færdigt}

\title{Printerlokaler}
\author{Sofie, Stolberg og Michael}

\begin{document}
\maketitle

\begin{roles}
\role{I}[NB] Instruktør
\role{M}[Malthe] Professor
\role{M}[Kristian] Professor
\role{M}[Anna M] Professor
\role{M}[Christine] Professor
\role{M}[Marcus] Professor
\end{roles}

\begin{props}
\prop{Kommer senere...}[]
\end{props}
  
\begin{sketch}
\says{F} To halvdårligt karikerede forelæsere kommer ind på scenen.

\says{M1} Lad os tænde computeren, så jeg kan vise dig den nye kattevideo.

\scene{De tænder computeren}

\says{M2} Hold da op hvor er den langsom.

\says{M1} Ja, godt det ikke er os, der er afhængige af de her computere.

\scene{Vindheks ruller forbi - skilt med 20 min senere.}

\says{M1} Godt, så lad os gå ind på YouPo... Jeg mener YouTube.

\says{M2} Årh, det andet kunne have været hordt - ikke sandt.

\says{M1} Jovist - min gode bjergfigur - men nu skal du se videoen.

\scene{Lyd af film afspiles}

\says{F} Forelæserne morede sig kosteligt.

\says{M1+2} Hahaha. Vi morer os!

\says{F} Bedst som de stod og morede sig, kom yderligere en virkelig dårligt karikeret forelæser på scenen, og udbrød:

\says{M3} Hvad sker der?!?

\says{M1} Kom og se denne her video.

\says{M3} Hvad er det for noget?

\says{M2} Det er lige så sjovt, som når en dårlig rus er til mundtlig eksamen.

\scene{Lyd af video}

\says{M1-3} Haha (de tager sig til låret)

\says{M3} Haha - lad os vise den til Ib.

\says{F} Ej - det tør vi ikke - Ib Madsen mangler nye læderbukser for tiden!

\says{M3} Haha - lad os vise den til Erik.

\says{F} Tilfældigvis kom Erik Kjær ind i lokalet.

\scene{M4 kommer på scenen.}

\says{M1} Hvem er det nu du er?

\says{M4} Jeg er den hemmelige institutleder...

\says{M3} Erik, du skal lige se denne video.

\says{M4} Det skal den hemmelige institutleder ikke sættes i forbindelse med - vi må slette søgehistorikken.

\says{M2} Hvordan gør man det.

\scene{Se kigger på hinanden under pinlig tavshed.}

\scene{Erik Kjær skubber langsom computeren på gulvet.}

\says{M4} Vi kalder det besparelser.

\says{M1} Men giver det mening kun at spare en computer væk.

\says{M4} Jeg gør som vi plejer - jeg sender den videre til Katja.
\end{sketch}

\end{document}
