\documentclass[a4paper,11pt]{article}

\usepackage{revy}
\usepackage[utf8]{inputenc}
\usepackage[T1]{fontenc}
\usepackage[danish]{babel}

\revyname{MatematikRevy}
\revyyear{2014}
\version{1.0}
\eta{$0.5$ minutter}
\status{Udkast færdigt}

\title{Jydesketch 1}
\author{Jakob og Shake}

\begin{document}
\maketitle

\begin{roles}
\role{I}[Shake] Instruktør
\role{J}[Ann G] Jyde
\end{roles}

\begin{props}
\prop{Kommer senere...}[]
\end{props}

\begin{sketch}
\scene{Jysk ‘instråktor’ kommer ind, har tavle med graf.}

\says{J} Mojn. A hedder Mariii-a, å a kommer fra Aarhus av. A skal være jeres instråktor. Idav ska’ vi snakk’ om skjæringssætningen. Hvis du har vær’t hæ’r.

\scene{J peger under x-aksen.}

\says{J} Og du har vær’t hæ’r.

\scene{J peger over x-aksen.}

\says{J} Så har du satane'me å vær’t hæ’r.

\scene{J peger på skæring med x-aksen.}

\says{J} Og så’n er det.
\end{sketch}

\end{document}
