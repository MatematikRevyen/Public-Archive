\documentclass[a4paper,11pt]{article}

\usepackage{revy}
\usepackage[utf8]{inputenc}
\usepackage[T1]{fontenc}
\usepackage[danish]{babel}

\revyname{MatematikRevy}
\revyyear{2014}
\version{1.0}
\eta{$2.5$ minutter}
\status{Udkast færdigt}

\title{Jacobis Vidner}
\author{Stolberg}

\begin{document}
\maketitle

\begin{roles}
\role{I}[Shake] Instruktør
\role{J}[Malthe] Jacobis Vidne - missionær
\role{M}[Michael] Matematiker
\end{roles}

\begin{props}
\prop{Kommer senere...}[]
\end{props}
  
\begin{sketch}
\scene{J ligner en religiøs. J kigger på et papir.}

\says{J} Nå, den næste er åbenbart professer i matematik.

\scene{J kigger drømmende op.}

\says{J (fortsat)} Mon ikke jeg kan få åbnet hans hjerte for Guds ord.

\scene{J ringer på. M åbner døren.}

\says{J (fortsat, entusiastisk)} Har du et øjeblik til at tale om den endegyldige sandhed?

\says{M (skeptisk)} Jooh, ud fra hvilket aksiomsystem?

\says{J (tøvende)} Øhmm... Altså, jeg kommer fra Jacobis Vidner.

\says{J (fortsat, entusiastisk)} Jesus var jo en storslået person. I en alder af 12 år blev han fundet i templet, hvor \emph{han} belærte \emph{præsterne}.

\says{M (skeptisk)} Han lyder lidt som Gauss... Var Jesus også sådan et arrogant røvhul?

\scene{J bliver forskrækket. J slår korsets tegn.}

\says{J} Nej, nej, nej... Jesu prædikede om at elske sin næste, selv sine fjender!

\says{M (meget skeptisk)} Hvad? også dataloger og fysikere?

\says{J (entusiastisk)} Ja, ja, Jesus elsker alle!

\says{M} Det lyder som noget sludder.

\scene{M begynder at lukke døren i. J stopper den med sin fod og skifter emne.}

\says{J (desperat)} Vent, har du hørt historien om Kain og Abel?

\says{M (irriteret, konfronterende)} Klein og Abel? Nej, hør nu, Abel døde tyve år før Klein overhoved blev født, hvad skulle det være for en historie?

\says{J (forvirret, lidt skræmt)} Nej, øhm... hvad med historien om Moses på Bjerget så?

\says{M (skeptisk, nysgerrig)} Hmm... Nej, den har jeg ikke hørt.

\scene{J bliver lettet.}

\says{J (dramatisk)} Nu skal du høre! Moses var en stor profet, han befriede det israelske folk fra Faraoen og førte dem over det røde hav! Og så... så vandrede han op på på toppen af Sinai-bjerget, hvor han var i 40 dage og 40 nætter! Da han endelig kom ned medbragte han de 10 grundlæggende regler, som alle skulle følge, og...

\says{M (vred)} Nej nu må det stoppe! ZFC blev formuleret af Zermelo og Fraenkel, ikke en eller anden Moses! Og desuden indeholder den kun 9 aksiomer!

\scene{J bliver forskrækket og flygter, mens M råber og skælder.}

\says{M (råber)} Ja, smut med dig! Og lad være med at komme igen før du har fået lidt forstand på matematik!

\scene{Lys ned. Tekst på skærm: "3 år og en bachelor i matematik senere". Lys op. J kommer trækkende med en vendetavle og ringer på.}

\says{M (først venligt så vredt)} Goddag... Hov er det nu dig med røverhistorierne igen! Hvad vil du?

\says{J (bestemt)} Jeg er kommet for at fortælle dig lidt om Jesus! Ser du...

\scene{J vender tavlen, graf kommer til syne.}

\says{J (fortsat)} Jesus liv kan rimeligvis beskrives som et tredjegradspolynomium. Han blev født til $t = 0$. Som du ser her opnår Jesus sine to lokale maksima lige omkring bjergprædiken og den sidste nadver. Herefter falder funktionsværdien kraftigt, og Jesus når sågar under $x$-aksen i tre dage, inden han igen bliver positiv. Her vandrer han så på jorden i 40 dage, inden han til slut går mod uendeligt.

\says{M (imponeret)} Interessant! Se, det giver jo mening! Hvad er så dette minimum?

\says{J (undskyldende)} Jaa, når man kan lave vand til vin, er man næsten dømt til at have nogle voldsomme tømmermænd en gang imellem...

\says{M (fornøjet)} Nååh, høhø, ja naturligvis.

\says{J (prøvende)} Nå, kunne du måske tænke dig at blive en del af vores menighed?

\says{M (overbevist)} Ja! Men der er bare et problem.

\scene{M stiller sig tættere på J.}

\says{M (stille)} Ser du, jeg har ikke altid været en særlig god person... Jeg har f.eks. divideret med $x$, uden at tage højde for om $x = 0$.

\scene{J begynder at forklare på tavle.}

\says{J (overbærende)} Bare rolig. Hør her: Lad $x(t)$ være en vilkårlig person til tiden $t$. Da gælder der, at for alle $t$ er $x(t)$ element i $G$, hvor $G$ er guds rige. Faktisk kan du forestille dig gudsrige som en familie $\{G_\alpha\}_\alpha \in U$ - Hvor $\alpha$ kan være et hvilket som helst element fra grundmængden! Vi er alle en del af guds rige!

\says{M (meget lettet)} Åh, det er jeg lettet over at høre! Jeg føler mig så glad nu! Hør, hvad med at jeg tager kage med til jeres næste forsamling i menigheden!

\scene{J griner lidt.}

\says{J (seriøs)} Haha... Vær ikke fjollet, vi må da selvfølgelig ikke spise kage!

\scene{Stilhed. M fortrækker sig på et øjeblik til en iskold vred mine og smækker døren i hovedet på J. Lys ned.}
\end{sketch}

\end{document}
