\alignment{
  \begin{center}
  \begin{Large}
  \textbf{Aksiom 0}\\[10pt]
  Den tomme m�ngde tilh�rer mig\\[5pt]
  \end{Large}
  Der er ikke andre tomme m�ngder end min tomme m�ngde
  \end{center}
}\newpage

\alignment{
  \begin{center}
  \begin{Large}
  \textbf{Aksiom 1}\\[10pt]
  Man kan kalde den tomme m�ngde, hvad man vil\\[5pt]
  \end{Large}
  For eksempel 'tre' eller 'tre plus fem'
  \end{center}
}\newpage

\alignment{
  \begin{center}
  \begin{Large}
  \textbf{Aksiom 2}\\[10pt]
  Hen til kommoden og tilbage igen\\[5pt]
  \end{Large}
  Det er da lige langt
  \end{center}
}\newpage

\alignment{
  \begin{center}
  \begin{Large}
  \textbf{Aksiom $\pi$}\\[10pt]
  $\pi \neq 3$
  \end{Large}
  \end{center}
}\newpage

\alignment{
  \begin{center}
  \begin{Large}
  \textbf{Aksiom 4}\\[10pt]
  \end{Large}
  \begin{large}
  Krokodillen gaber altid mod den m�ngde, der er st�rst
  \end{large}
  \end{center}
}\newpage

\alignment{
  \begin{center}
  \begin{Large}
  \textbf{Aksiom 5}\\[10pt]
  \end{Large}
  Disse bud udg�r et konsistent og fuldst�ndigt aksiomsystem,\\
  og man kan s�gar udlede eksistensen af de hele tal ud fra det
  \end{center}
}\newpage

\alignment{
  \begin{center}
  \begin{Large}
  \textbf{Aksiom 8}\\[10pt]
  $6$ er syndigt, og ved induktion er $7$ ogs� syndigt,
  hvorfor disse udelades
  \end{Large}
  \end{center}
}\newpage

\begin{large}
\rule{0.1pt}{0.1pt}\vspace{1cm}

\noindent\textbf{Korollar}\\[10pt]
$42$ er et primtal, som produkt af syndige udeladte tal
\end{large}
\newpage

\alignment{
  \begin{center}
  \begin{Large}
  \textbf{Aksiom 9}\\[10pt]
  Det er ikke et sp�rgsm�l om 2\phantom{,}\\
  \phantom{det er et sp�rgsm�l om tro}
  \end{Large}
  \end{center}
}\newpage

\alignment{
  \begin{center}
  \begin{Large}
  \textbf{Aksiom 9}\\[10pt]
  Det er ikke et sp�rgsm�l om 2,\\
  det er et sp�rgsm�l om tro
  \end{Large}
  \end{center}
}\newpage
