\documentclass[a4paper,11pt]{article}

\usepackage{revy}
\usepackage[utf8]{inputenc}
\usepackage[T1]{fontenc}
\usepackage[danish]{babel}

\revyname{Matematik Revyen}
\revyyear{2015}
% HUSK AT OPDATERE VERSIONSNUMMER
\version{1.}
\eta{4 minutter}
\status{Skrevet}

\title{Inversmanden vender frem}
\author{Matematikrevyens Manusgruppe}

\begin{document}
\maketitle

\begin{roles}
\role{X}[William] Instruktør
\role{I}[Kristian] Inversmanden
\role{E}[Iris] Elskerinde
\end{roles}

\begin{props}
\prop{Rip/Rap/Rup kasket i et}
\end{props}

%\begin{mics}
%\mic{HS1}[] ???
%\end{mics}
  
\begin{sketch}
\scene{AV med 'inversmanden vender frem som overtekst - se mail, 3 billeder'}
\scene{Knalde lyde.}
\says{I} Sig jeg har en llle pik!
\says{E} Du har en stor pik!
\says{I} ÅåååhHhåååHH... Je..Je..Jeg går nu! ...  ÅååHHH!
\scene{lys op. AV med 'inversmanden vender frem' fjernes}
\act{De kommer ind på scenen. Inversmanden er iført underbukser og en skjorte, der står åben. Han sveder lidt. Hun har trusser på og en skjorte der lige dækker numsen.
}
\says{I} Åh, hvor var det bare ualmindelig klamt!
\says{E} Ja, det kunne jeg både høre og se på dig (med et sælsomt smil på læben). Du har da også været ude og træne dig siden sidst!
\says{I} \act{Manden ser ned ad sig selv og på sine arme}Ja, jeg er ret kvapset. (flexer lidt mere)... mine uvenner siger ofte, at jeg egentlig er tæske feminin!
\says{E} Skat, du skal altså vide, at jeg bare elsker dig så højt.
\says{I} Åh min kælling, du er så grim... at der findes et tidspunkt, hvor jeg ikke har lyst til at se på dig.
\says{E} Uh, du kan da også bare det der med ord!
\act{Inversmanden går hen til Nejni og dufter meget tydeligt til hende.
}
\says{I} Uuuuhhmm. Du stinker så grimt, din klamme so.
\says{E} Du er simpelthen så sød. Det er altså for galt, skat. Men skat, jeg havde faktisk noget jeg gerne ville tale med dig om.
\says{I} Ja, lad os hellere holde vores kæft.
\says{E} Du kan tro, de andre kvinder på arbejdet misunder mig, at jeg har en fyr, der er så nem at tale med. Jo altså, selvom jeg nyder hver en dag i dit selskab, og selvom jeg kan blive helt høj over, at det er dig, jeg får lov til at ligge i ske med hver nat, så tænkte jeg på om ikke vi kunne tale lidt om hvordan vores forhold udvikler sig?
\says{I} \act{Tager E på skulderen} Nej, nej, det kan jeg virkelig ikke se nogen grund til.
\says{E} Godt, fordi tillid er hjørnestenen i ethvert parforhold.
\says{I} Nej, det er jeg meget uenig i. Det går jo i forvejen lige så dårligt.
\says{E} Altså, jeg er helt seriøs.
\says{I} Jaja, jeg er også bedøvende ligeglad.
\says{E} Det er bare fordi, Molly og Lis henne på arbejdet… Ja, de venter sig begge.
\says{I} Nååå… Skat, du ved jo godt, at jeg hader børn. Disse små forbryderiske voldspsykopater…
\says{E} Jamen, det er jeg vældig glad for at høre. For ser du, jeg venter mig allerede… Der er faktisk kun otte uger til termin \act{skaf en oppustelig mave, der pustes op i det samme dette siges}.
\says{I} Hva?! Woow, det havde jeg lige set komme!
\says{E} Rolig nu, skat det skal nok løse sig altsammen!
\says{I} Hvorfor siger du det i så god tid? Jeg har jo slet ikke andre ting at tage mig til for tiden.
\says{E} Ej ej, du må altså ikke tage det på den måde.
\says{I} Nu vil jeg altså også have lov til at være glad.
\says{E} Øhm (halvstammende, kiggende ned på gulvet)... Jeg har altså også sagt mit job op, så der kan blive tid til den lille ny… Så hvis du kunne tage lidt overarbejde her… de næste… 18 år…
\act{I sukker}
\says{I} Åååh! Kvinder! I er altså nogen gange bare de klogeste at høre på. Har I nogensinde noget dumt at sige?
\says{E} Skat, sådan behøver du ikke at reagere.
\says{I} Jeg kommer nu!
\scene{Inversmanden går sin vej.}
\says{E} Ej. Undskyld skat. Ikke blive sur og tvær.
\scene{Inversmanden kommer ind på scene igen med en smuk buket blomster.
}
\says{I} Undskyld, Jeg er ked på det. Jeg mente det - så lad os slå en streg under det. 
\says{E} Jeg tilgiver dig, men på éen betingelse; kan du ikke lade vær med at sige det modsatte hele tiden.
\says{I} ..Jo, det vil jeg godt \act{smilende}
\scene{Lys ned}
\end{sketch}
\end{document}
