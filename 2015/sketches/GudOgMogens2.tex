\documentclass[a4paper,11pt]{article}

\usepackage{revy}
\usepackage[utf8]{inputenc}
\usepackage[T1]{fontenc}
\usepackage[danish]{babel}
\usepackage{amssymb}

\revyname{Matematik Revyen}
\revyyear{2015}
% HUSK AT OPDATERE VERSIONSNUMMER
\version{1.0}
\eta{2 minutter}
\status{Færdig}

\title{Gud og Mogens 2}
\author{Anna, William, Kristian og Michael}

\begin{document}
\maketitle

\begin{roles}
\role{X}[Jakob] Instruktør
\role{G}[Marianne] Gud
\role{F}[Stolberg] Fortæller
\role{M}[Emil] Mogens
\end{roles}

%\begin{props}
%\end{props}

%\begin{mics}
%\mic{HS1}[] ???
%\end{mics}
  
\begin{sketch}
%\scene{Lys op.}

\says{G} Moooogens! Kom lige herind. Altså jeg er sku ved at være lidt træt af de der mennesker.

\says{M} Ja okay, så jeg skal slå dem allesammen ihjel?

\says{G} Ej, Mogens. Det skal du ikke.

\says{M} Nåh. Skal jeg så slå alle børnene ihjel.

\says{G} Ej - vent lige. Der har været nogle misforståelser her på sidste. Først var der den med Abraham og Isaak.

\says{M} Ja; den tager jeg på min kappe. Men den fik vi heldigvis reddet i tide.

\says{G} Og så var der babelstårnet, som der ikke er så meget at gøre ved.

\says{M}[Optimistisk] Men jeg har købt det her kæmpe parti ordbøger jo. 

\says{G} Men det jeg skal til at sige nu, skal ikke blive en fiasko ligesom dem, så det er meget vigtigt at du hører godt efter hvad jeg siger.

\says{M} O-oka-y.

\says{G} Menneskene var været syndige mod hinanden og mod mig. Og de skal have et lille signal om at de har været syndige.

\says{M} Og det skal ikke være at vi slår dem ihjel.

\says{G} Nej, det skal det ikke.

\says{M} Hvad så med at vi giver dem lidt dårligt vejr?

\says{G} Ja, midt i badesæsonen, det lyder okay.

\says{M} \act{Bevæger sig mod scene udgangen} Got it. 

\says{F} Og da regnede det i 40 dage og 40 nætter og alle døde, pånær Noah, hans familie og en hulens masse dyr.

\says{G} Moooooogens!



%\scene{Lys ned.}
\end{sketch}
\end{document}

