%        File: Midterpolemik.tex
%     Created: Søn Nov 08 10:00 am 2015 C
% Last Change: Søn Nov 08 10:00 am 2015 C
%
\documentclass[a4paper,11pt]{article}

\usepackage{revy}
\usepackage[utf8]{inputenc}
\usepackage[T1]{fontenc}
\usepackage[danish]{babel}

\revyname{MatematikRevy}
\revyyear{2015}
\version{1.0}
\eta{Ved ikke}
\status{Udkast færdigt}

\title{Midterpolemik}
\author{Michael, Kristian, Anna og andre...}


\begin{document}
\maketitle

\begin{roles}
\role{I}[Shake] Instruktør
\role{C}[Michael] Clement Kjærsgaard
\role{K}[Rolf] Kristian Thulesen Dahl
\role{U}[Jasper] Uffe Elbæk
\role{J}[Iris] Johanne Schmidt-Nielsen
\end{roles}

\begin{props}
\prop{Talepulte}[]
\prop{Danskvand}[]
\prop{Kommer senere...}[]
\end{props}

\begin{sketch}
\says{C} I aften skal vi endevende den danske politiske midte; for i en tid hvor alle partier gerne vil stemples som midterpartier får man lyst til at spørge ind til, hvorfor der i det hele taget er så mange selverklærede midterpartier - og kan de virkeligt allesammen befinde sig på midten?
\says{C}[Henvendt til K] Kristian Thulesen Dahl, hvorfor tror du, at der er så mange partier på midten?
\says{T} Jamen, det er jo vigtigt at være dér, hvor danskerne er. Og i øvrigt er danskerne jo Guds udvalgte folk, så det er da naturligt at holde sig i centrum.
\says{C} Men Kristian - hvordan kan du tro at der er plads til så mange på noget så småt som en midte - man skulle da mene, at der kun kan være én ting på midten.
\says{T} Ja, nu ved jeg ikke hvor du har gået i skole, men der hvor jeg har gået i skole i Jylland ude blandt de rigtige danskere, der ved vi nu godt, at der kan være flere ting på en midte… Ja, tag nu for eksempel ækvator - der ligger da flere lande på ækvator, og der ligger de jo ganske godt, så skal vi ikke lade dem blive i nærområdet.
\says{C} Ja ok, og Uffe Elbæk, hvad kommer du så til at tænke på, når vi taler om, at mange opholder sig på denne såkaldte midte?
\says{U} Jamen, først vil jeg vil jeg lige sige, at jeg synes faktisk det er vældig dejligt at høre Kristian her i aften anerkende at ækvator findes, for det er som om, at der alt for få af os på Christiansborg, der tager ud i virkeligheden, og ser hvad der rører sig. Tag nu for eksempel en af mine a-a-alt for få mærkesager: Klimaforandringerne...
\says{C} Kan vi ikke lige prøve at holde fokus i debatten?
\says{U} Jamen, det er da lige det jeg prøver på, søde Martin! Det er jo dig, der prøver at snakke udenom klimaforandringerne! Jeg mener at det er tid til forandring.
\says{C} Nu var fokus jo oprindeligt på spørgsmålet om midterpolitik - og i øvrigt har jeg altså også lige behov for at høre min egen stemme lidt mere, så lad mig lige gentage, at vi altså er i gang med at gennemanalysere begrebet “midten” - og lad mig i den forbindelse i øvrigt nævne denne rapport om emnet, som en praktikant umiddelbart inden udsendelsen har stukket i hænderne på mig og som jeg endnu har haft lyst til at læse - og den siger, at nogle danskere engang har overvejet om de på et senere tidspunkt måske kunne få lyst til at overveje at bevæge sig hen mod midten. Og Johanne Schmidt-Nielsen i ligestillingens hellige navn må jeg jo nok også lige høre om en kvinde rent faktisk har holdninger?
\says{J} Masser! Altså, vi kvinder består nærmest ikke af andet - og for at være på den sikre side har vi da også gerne flere holdninger til de samme sager. For eksempel synes vi ikke bare vi kan nøjes med gode midterpartier - vi mener både det vigtigt at have gode for- og bagpartier. Det er vigtigt, at der er plads til mangfoldighed.
\says{T} Ja, det er meget vigtigt, at vi er mangfoldige - når blot vi er det på den SAMME måde! Nemlig den danske måde!
\says{U} Det er også vigtigt at huske, at vi fortsat skal blive bedre til at eksportere viden, idéer og andre diffuse begreber.
\says{C} Forskning viser, at der plads til op til to midterpartier - men det harmonerer ikke med det nuværende politiske billede. Så hvilke to partier tror du det vil være om fem år Thulle?
\says{T} Nu vil jeg ikke udtale mig om den konkrete sag, men helt generelt kan jeg sige at det bliver de 2 partier, som det danske folk finder bedst og mest dansk.
\says{C} En anden undersøgelse fra Nordkorea viser at der kun kan være et parti, og så er det per definition på midten. Hvem bliver det om 5 år, Johanne Schmidt-Nielsen?
\says{J} Nu vil jeg ikke udtale mig for meget om det generelt, men i det her konkrete tilfælde vil jeg sige at det for det første kunne være rart at tage en fordomsfri diskussion om Nordkorea. Rundt omkring i verden ser man at når der kun er 1 parti, er det et ofte rødt parti, som fx i Nordkorea. Så derfor er det kun rimeligt at danmark står sammen om det eneste ambitiøse enhedsskabende revolutionære parti - Enhedslisten.
\says{C} Det er en vældig interessant, men dog fuldkommen irrelevant pointe, du rejser dér. Og lad mig for en god ordens skyld understrege, at jeg er helt og aldeles upartisk her.
\says{U} Men hvad er midten overhovedet? Jeg synes at det er en typisk gammeldags Christiansborg-tænkning med at man skal mødes på midten. Hvad med at man skal mødes på på siden.
\says{C} What?
\says{J} What?
\says{T} Dansk hvad?
\says{C} Hvad mener du egentlig med det, Uffe?
\says{U} Det ved jeg jo ikke selv endnu. Det har jeg jo ikke gennemtænkt. Det kunne måske være at man vælger et parti, som ikke er et midterparti, og så bare gør hvad de siger. På bedste demokratiske vis. Alternativt kan vi bare slå partierne sammen.
\says{J} Perfekt, og så kan vi kalde det Alternativ Dansk Enhed. Det kunne i snit blive det mest midterlige midterparti.
\says{T} Det lyder godt, og så tager jeg i øvrigt lige en tår af min danskvand.
\says{U} *uffe lyde* Det er jo crazy, det her.
\end{sketch}

\textit{Lys ned. Alle griner!}

\end{document}




