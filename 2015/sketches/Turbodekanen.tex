\documentclass[a4paper,11pt]{article}

\usepackage{revy}
\usepackage[utf8]{inputenc}
\usepackage[T1]{fontenc}
\usepackage[danish]{babel}

\revyname{Matematik Revyen}
\revyyear{2015}
% HUSK AT OPDATERE VERSIONSNUMMER
\version{1.0}
\eta{$2.5$ minutter}
\status{Færdig}

\title{Turbo Dekanen}
\author{William, Anno, Stolberg \& Nicolas}

\begin{document}
\maketitle

\begin{roles}
\role{X}[Alexander] Instruktør
\role{T}[Sofie] Turbo-Dekan
\role{P}[Jasper] Professor
\role{NP1}[Johanna] Nummerpige 1
\role{NP2}[Iris] Nummerpige 2
\role{NP3}[AnnG] Nummerpige 3
\end{roles}

\begin{props}
\prop{Matintro skilt}
\prop{Koman skilt}
\prop{Algebra skilt}
\prop{Topologi skilt}
\prop{Linalg skilt}
\end{props}

%\begin{mics}
%\mic{HS1}[] ???
%\end{mics}
  
\begin{sketch}
\says{T} Hej allesammen, i kender mig, i elsker mig, det er mig dekanen!! Eller som jeg har valgt at kalde mig i under fremdrifts-reformen: TURBO-DEKANEN!!!
\scene{Storskærmen: TURBO-DEKAN!!!, med lykkehjuls tema}
\says{T} Vi har jo i et stykke tid sparet på KUA, lokalebookning, eftertilmelding og print. \\
I den forbindelse har jeg brugt en masse - økonomiske! - resurser på "rebranding"! Og jeg har glædet mig meget til at præsentere det nye "brand".
Vi vil gerne "lable" på en måde, der sender de rigtige "signals" til de studerende. \\
Som et godt eksempel ændrer vi Vandrehallens navn til Løbehallen og svalegangen bliver nu til falkegangen \act{kort pause, hvor T ser meget selvtiltreds ud}.
Det giver de studerende idéen om at man skal HURTIGT igennem studiet.
\says{P} Ej, hør nu! Vi må lige tage det roligt og tænke på den historie, der ligger til grund for den vandrehal vi har i dag.
\says{T} \act{Dybt suk} Den diskussion har vi SLET ikke "time for". Der er nemlig mange andre "rebrands". Fx hedder det ikke længere juleferie. Men januarforberedelse, og i stedet for at holde sommerferie, holder man nu blok 5 og 6. \\
Men vi har slet ikke tid til alt det. \\
Brand-renewals, der er relevante for jer er at  Institut for matematiske fag “MATH” bliver til Institut for fem-årige matematikuddannelser “FAST”.  -Ja hverken den gamle eller den nye forkortelse passer til navnet. \\
Men det stopper ikke der, for nu kommer det til at hedde det “Det io- og naturvidenskabelige fakultet” med sloganet ‘vi har sparet b’et væk, så du kan spare endnu mere (tid på din uddannelse)’. \\
Hold kæft hvor er jeg god til at bruge penge på de rigtige ting, jeg fortjener vist en bonus- sagt i al beskedenhed selvfølgelig. Og som vi siger i administrationen: “Man kan ikke bruge for mange penge på administration.”

\says{P} Hov, stop vent.

\says{T} \act{afbrydende} Ja, jeg ved lige hvad du tænker! - det er godt med ligegyldige ændringer i ting, Men hvad er relevant for mig?.\\
Fx hvad kommer de nye kurser til at hedde?
Turbodekanen er "ready with the answers!". \\
MatIntro bliver til MatOutro. \act{NP1 kommer ind med Matintro skilt i det der siges matintro}\\
KomAn hedder nu KomVidere. \act{NP2 kommer ind med Koman skilt i det der siges Koman} \\
Algebra er blevet til knap-så-meget-gebra.\act{NP3 kommer ind med Algebra skilt i det der siges Algebra} \\
Dette skyldes at VI på administrationen har fundet ud af at jo mindre pensum er, jo hurtige kommer de studerende igennem!
\says{P} Hvad så med Algebra 2 og Algebra 3
\says{T} Ja det må i matematikere så kunne "regne ud", du ved "rain it out". Uhh og Topologi har forresten også skriftet navn, det findes nu som Turbologi. \act{NP1 kommer ind med Topologi skilt når der siges topologi}\\
\scene{P springer så op og udbryder}
\says{P} Ej nu må det altså stoppe. Sku’ lineær algebra så hedde eksponentiel algebra?!
\says{T} Nej, nej, nej! Lytter i overhoved efter? \act{NP2 kommer ind her med LinAlg skiltet} Det bliver eksponentiel knap-så-meget-gebra, og som I ved, så har det bedste forsvar altid været et angreb - og dette kommer også til at gælde for de nye bachelorANGREB.

\says{P} Hvad med logik?
\says{T} \act{Fortvivlet} Logik... øhhh... hmmm... Det streger vi helt - det strider imod ånden i fremdriftsreformen.
\end{sketch}
\end{document}

%%% Local Variables: 
%%% mode: latex
%%% TeX-master: t
%%% End: 

