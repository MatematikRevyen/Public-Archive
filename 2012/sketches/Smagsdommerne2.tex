\documentclass[a4paper,11pt]{article}

\usepackage{revy}
\usepackage[utf8]{inputenc}
\usepackage[T1]{fontenc}
\usepackage[danish]{babel}


\revyname{MatematikRevy}
\revyyear{2012}
% HUSK AT OPDATERE VERSIONSNUMMER
\version{1.0}
\eta{3 minutter}
\status{Skal Redigeres}

\title{Smagsdommerne}
\author{Manusgruppen}

\begin{document}
\maketitle

\begin{roles}
\role{I}[KOEK] Instruktør
\role{V}[Emil] Adrian Hughes
\role{G}[Kirsten] Smagsdommer
\role{S}[Anne G] Smagsdommer
\role{K}[William] Smagsdommer
\end{roles}

\begin{props}
\prop{Rekvisit}[Person, der skaffer]
\end{props}


\begin{sketch}
\says{A} I denne omgang af smagsdommerne overværede vi en nyfortolkning af den dansk-jydske julekalender - The Julekalender; hvor russerne kæmpede en kamp for at færdigøre deres MatIntro-aflevering, og instruktorerne, isært den synskiftende karakter Benny prøver at forpurer de unge studerendes agenda.
\says{L} Jeg mener persontlig at Gertrud havde en særdeles yndig (farve) kjole på.
\says{M} (Noget klogt om farver)
\says{L} Jaja, men hvad laver hun også med ham Oluf? Han er tydeligvis vokset op i en fattig familie.
\says{P} Ja interessant finder jeg det, at så livsbekraftende en kvinde er gift med en karakter som Oluf. Manden er jo tydeligvis en slave af det etablerede samfund, i det han nægter, at se problemerne i øjnene og konstant incisterer på, at alt bare er dæjligt.
\says{L} Ham Benny, han var fuld af gode ideer. Jeg synes virkelig, at han brillierede i sin rolle.
\says{M} Jeg tænker, at Bennys udvikling gennem stykket viser den alkoholisme, vi oplever ved juletid - især omkring første december tendencerer især ungdommen til at intage faretruende mængder af den gyldne, euforiserende, betagende og velsmagende dame.
\says{L} Jeg synes nu egentligt, at ungdommen har fat i den lange ende - skål!
\scene{Tæppe}

\end{sketch}
\end{document}
