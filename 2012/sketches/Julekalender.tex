\documentclass[a4paper,11pt]{article}

\usepackage{revy}
\usepackage[utf8]{inputenc}
\usepackage[T1]{fontenc}
\usepackage[danish]{babel}

\revyname{MatematikRevy}
\revyyear{2012}
% HUSK AT OPDATERE VERSIONSNUMMER
% UNDLAD AT SKRIVE I TEMPLATE.TEX - KOPIÉR OG OMDØB I STEDET FOR
\version{$3.0$}
\eta{5 minutter}
\status{Færdig}

\title{The Julekalender}
\author{Diverse}

\begin{document}
\maketitle

\begin{roles}
    \role{I}[Jakob] Instruktør 
    \role{Fo}[Ada]Fortæller (Voice-over)
	\role{Ge}[Lise] Getrud
	\role{Ol}[JOEP] Oluf
	\role{Be}[Soeren] Benny
	\role{Ha}[Kasper Brandt] Hansi
	\role{Fr}[Freja] Fritz
	\role{Gü}[Mini Strunge] Günther
	\role{N1}[Jenny] Ninja
	\role{N2}[Jasmin] Ninja
\end{roles}


\begin{props}
\prop{To borde}
\prop{Seks stole}
\prop{Kalkulus}
\prop{Stak afleveringer}
\prop{To kaffekopper}
\prop{Cola}
\prop{Fez}
\prop{Kugleramme}
\prop{Papkat}				
\end{props}

\begin{sketch}

\scene{Det er mørkt på scenen. Der kører en film i stil med åbningen af hvert The Julekalender afsnit under fortællerens tale.}

%\says{Fo} I mange timer har der siddet 3 russer, Fritz, Günther og Hansi,  og arbejdet på en MatIntro-opgave. Med sig har de den store lærebog, hvori de kan finde svar på næsten alt. Men de kan ikke komme hjem, før at de har afleveret. Hvad værre er – de kan ikke få forbindelse til Eduroam.\\
%Alt dette ved Intruktor Oluf, hans kæreste Gertrud og deres kat Emil intet om. De sidder blot og retter afleveringer. Benny, deres mystiske medintruktor sidder med et svarark og retter ligeledes. Russerne ved, at han har svararket, men får de selv fundet frem til svarene? \\
%Det er en mørk og stormfuld aften.

\scene{Lys op på bordet til venstre på scenen, hvor russerne sidder.}
\says{Ha} Den her MatIntro-aflevering er da godt nok svær.
\says{Fr} Hvordan dælen laver vi den mon?
\says{Gü} Lad os kigge i bogen.
\says{Fr} Lige hvad jeg skulle til at sige.
\scene{Günter kigger i Kalkulus}
\says{Gü} Ah shit, det er på norsk.
\says{Fr} Giv den da til mig. Lad mig se. Determinant, differentialkvotient, din mor, distributivitet! Her er det. S. 123.
\says{Ha}[Afbryder] Hvad med Lagrange Restled?
\says{Fr} Hov, der er en fed sætning. Måske kan vi bruge den i en anden opgave. \act{mod Günther} Laver du lige mapledelen imens?
\says{Ha} Hvorfor er det altid mig?
\says{Fr} Fordi, det er dig, der er dummest og har datalogi som sidefag.


\scene{Lys ned på russerne, lys op på bordet til venstre, hvor Benny og Oluf sidder. Benny sidder og retter meget koncentreret og laver de karakteristiske bevægelser med armene.}
\says{Be} Håhrr. Amatørrusser!
\scene{Gertrud kommer ind med en stak afleveringer.}
\says{Ge} Oluf, a' står li' me' de her opgaver? De laver simpelthen så mange fejl ik'. De bruger ik' engan' L'Hôpital, når de har med nul over nul over nul over nul over n...
\says{Ol}[Afbryder] Ja det er bore dæjligt
\says{Ge}[Fortsætter] nul over nul over nul over nul..
\says{Ol}[Afbryder meget bestemt] A' sa': Det' bor' dæjligt.
\says{Be} Ja nu hedder det jo Lopital..
\scene{Gertrud og Oluf ser underligt på Benny.}
\says{Be} Ja altså Lop-ii-taal, ikk'? (mimer et loop). Det er Fransk, ikk'? Paris, ikk'? Toast, ikk'?
\says{Ol} Ja, det er bore dæjligt...
\scene{Benny sidder og læser koncentreret idet han lavmælt siger}
\says{Be} Ahhr, hvordan man dumper studerende, der er irriterende.
\says{Ol} Du ø' Benny, sku' vi ik' li' ha' æ' rom å' cola ti' æ' rettelser?
\says{Be} Ja øhh, nu drikker jeg ikk' spiritus, vel?
\scene{Oluf skænker imens han siger}
\says{Ol} Jo, jo, en lil' cola ka' man vel alti' drik'
\says{Be} Jeg har noget med maven, ikk'?
\scene{Oluf snupper sin kop og siger}
\says{Ol} Skål, du Benny.

\scene{Vi når lige at se dem skåle. Lys ned på intruktorer, lys op på russerne.}
\says{Fr}[Glad] Nu har vi den.
\says{Gü}[Skuffet] Nej ikke alligevel.
\says{Ha} Øv.

\scene{Lys ned på russer, lys op på intruktorer.}
\says{Ol} Skål igen, Benny!
\scene{De skåler, drikker og Benny transformeres nu til en datalog. Den samme lyd som kommer i serien når Benny transformeres, lyder}
\says{Be} Ahr! Vi skal lige tælle point op, ikk'? 
\says{Ol} Jo, det var li’ godt grov.
\says{Ge} Oluf, ved du hvad jeg ønsker mig i julegave? 
\says{Ol} Nej.
\says{Ge} Jeg ønsker mig sådan en TI 89+ lommeregner. Den kan både dividere og multiplicere og addere og substrahere. Hva' ønsker du dig Oluf?
\says{Ol} Ja, jeg ønsker mig vel en pakke kridt.
\says{Ge} Det ønsker du dig jo hvert år. Ku' du ikke prøve at finde på noget nyt?
\says{Ol}[sidder meget eftertænksom og siger så]  'a, så kunne jeg måske ønske mig en tavlesvamp.
\says{Be}[Snupper kuglepen og retter løs] Åhhr! Trækker 100 point. Hold kææft det ondt! \act{Slynger hovedet rundt ligesom i afsnit 19. Stopper op for en kort stund og ser undrende ud} 100 point?
\scene{Benny springer straks op hen til sin kuffert åbner den og snupper kuglerammen og begynder at tælle 100 point.}
\says{Be}[Meget ophidset mens han ryster kuglerammen] Hold kæft, det' det hele, man! Aaaghr! \act{snupper colaen og tager en stor tår. Og slutter af med at sige} AAAAARGH.
\says{Ge} Hvaa', ser du ik' lidt skidt u', Benny?
\says{Be} Nej, jeg har det fint, ikk'?
\says{Ol}[Eftertænksomt] Nu ved jeg ikke om det virker, men jeg ku' jo også prøve at ønske mig sådan en fancy kridtholder.
\says{Be} Jeg kom lige til at tænke på en joke, ikk'. Det var en matematiker, en fysiker og en biolog ikke. Jo og de var til eksamen i Matintro, ikk'? Så går fysikeren op. Banke, banke på. Så siger fysikeren, "Ja, jeg' fra fysik, ikk'". Så siger eksamensvagten, eller... hvordan var det nu lige den var...

\scene{Lys ned på intruktorer, lys op på russer.}
\says{Gü}[Glad] Nu har vi den.
\says{Ha}[Skuffet] Nej ikke alligevel.
\says{Fr} Øv.

\scene{Lys ned på russer, lys op på intruktorer.}
Benny og Gertrud sidder og skraldgriner mens Oluf bare sidder og løfter øjnbrynene og siger
\says{Ol} Det var li' godt grov.
\scene{Lys ned.}

%\says{Fo} Får russerne nogensinde lavet deres aflevering? Hvor gammel bliver en ræv? Og Hvordan slutter man sådan en sketch her af?  Følg med næste år og få svaret.

\scene{Film afslutter}

\end{sketch}
\end{document}