\documentclass[a4paper,11pt]{article}

\usepackage{revy}
\usepackage[utf8]{inputenc}
\usepackage[T1]{fontenc}
\usepackage[danish]{babel}


\revyname{MatematikRevy}
\revyyear{2012}
% HUSK AT OPDATERE VERSIONSNUMMER
\version{1.0}
\eta{3 minutter}
\status{Skal Redigeres}

\title{Smagsdommerne}
\author{Manusgruppen}

\begin{document}
\maketitle

\begin{roles}
\role{I}[KOEK] Instruktør
\role{A}[Emil] Adrian Hughes
\role{M}[Kirsten] Smagsdommer
\role{L}[Anne G] Smagsdommer
\role{P}[William] Smagsdommer
\end{roles}

\begin{props}
\prop{Rekvisit}[Person, der skaffer]
\end{props}


\begin{sketch}
\says{A} Velkommen til Smagsdommerne, hvor vi i denne uge skal tage et kig på novellesamlingen Kalkulus 3. udgave af den veletablerede norske forfatter Tom Lindstrøm, der blandt andet står bag bøger som Kalkulus 1. udgave og Kalkulus 2. udgave. Lad os her se en kort introduktion til bogen.
\scene{Kort intro film vises}
\says{A} Og med os i studiet i dag har vi Leonora Christina von Schou - privat kundstsamler og hesteopdrætter fra Holte; dernæst har vi Cand.mag i moderne kundst, litteratur, teater og dans ved Syddansk Universitet...
\says{M} Og ballettens historie.
\says{A} Margrethe Juliane Hertz af Marienburg. Sidst men ikke mindst har vi Poul Gunnar, professor emeritus i kundst ved Københavns Universitet.
\says{A} Og Margrethe - hvad mener du om dette kundstværk?
\says{G} Hvis jeg ser bort fra, at jeg ikke kan finde den store sammenhæng mellem alle historierne, så er jeg selv personligt utroligt glad for novellen: "Historisk Epistel: Fra Kaniner til Kaos", hvor en række af karakterer trækker en form for gyldent snit gennem handlingen. Man føler efter at have læst novellen, at man er endt i den rette placering i det store billede. Jeg forstår dog ikke brugen af de ikke-latinske bogstaver, såsom phi. Jeg er cand.mag i mange ting, men ikke i phi.
\scene{Alle griner}
\says{A}[Til K] Hvad med dem, føler de dem også rigtigt placeret?
\says{K} Nej, tvært imod føler jeg mig korrekt anbragt inden for rammerne. Men jeg vil jo mene at novellen "Funktionsfølger" er et fuldstændigt værk. Dette kommer specielt til udtryk i figuren Taylor's uendelige udvikling. Forfatteren får skabt nogle minimalistiske karakterer. Specielt var jeg dybt betaget figurerne $\varepsilon$ og $\delta$ der jo er uløseligt forbunde, og deres fælles gang mod det rene ingenting.
\says{S} Jeg tror her, at du tænker på kapitlet "Konvergens af følger", hvor Tom har et utroligt kontinuert flow. Man må sige at ordene kommer i en - lind strøm. Han får virkelig figurerne til at nærme sig hinanden. Og handlingen er jo fascinerende velkalkuleret.”
\says{K} Jeg må sige du har fuldstændig ret. Også i novellen "Sandsynligheder" er der helt klart tænkt over fordelingen af indholdet; trods det er skrevet i et utroligt diskret sprog. Han forstår, at vælge ligeligt mellem disse, sådan lidt tilfældige karakterer.
\says{A} I novellen "Antiderivasion" er, der ligesom et negativt syn på adskillelse der foregår i den her alternative verden hvor alt bevæger sig i modsat retning. \act{mod G} Hvad mener du om denne differentation.
\says{G} Ja - igen viser forfatteren sig fra sin stærke side, ved nogle utroligt velintegrerede karakterer. Der er en passende mængde af ligeheder og på samme tid uligheder mellem karaktererne. Han har også et delvist ubestemt udtryk. Den mængde af elementer, han formår at beskrive med så mange ensbetydende påstande er løseligt forenklende. Og jeg er uendeligt imponeret af hans eksemplificerende brug af minimale stavefejl. I det store hele skærer handlingsforløbende hinanden omkring en form for center i det koordinerede system af noveller.
\says{S} Jeg kunne ikke have sagt det bedre selv.
\says{A} Med denne note i mente må vi desværre slutte her og trække streg under resultatet. Men følg med i næste uge hvor vi anmelder den spændende nye forfatter Euclids debut roman "Elementer".
\scene{Lys ned}
\says{F} Er der ikke nogen, der har et arbejde til mig?
\scene{Tæppe}

\end{sketch}
\end{document}
