\documentclass[a4paper,11pt]{article}

\usepackage{revy}
\usepackage[utf8]{inputenc}
\usepackage[T1]{fontenc}
\usepackage[danish]{babel}


\revyname{MatematikRevy}
\revyyear{2012}
% HUSK AT OPDATERE VERSIONSNUMMER
\version{1.0}
\eta{3 minutter}
\status{Skal Redigeres}

\title{Smagsdommerne}
\author{Manusgruppen}

\begin{document}
\maketitle

\begin{roles}
\role{I}[KOEK] Instruktør
\role{V}[Emil] Adrian Hughes
\role{M}[Kirsten] Smagsdommer
\role{L}[Anne G] Smagsdommer
\role{P}[William] Smagsdommer
\end{roles}

\begin{props}
\prop{Mini bord}[Person, der skaffer]
\prop{Karaffel og glas}[Person, der skaffer]
\end{props}


\begin{sketch}
\says{A} Til slut i programmet skal vi berøre KUs seneste kundstværk - Matematikrevyen. Produceret og opført af matematikstudiets rørende og forførende kundstnerkolektiv.
\says{L} Ja, jeg synes jo, at texnikken gjorde et særdeles ekstraordinært arbejde. Men jeg forstod aldrig hvem ham styrmand Karlsen var?
\says{M} Bandet satte også hyldest til nogle rigtige klassikere. Dog må man sige, at det også involverede en del børnesange.
\says{P} Ja, men det passede jo også ret godt til sangerne. De virkede ikke som de ejede det store tallent. Nogle gange sang de alt for lavt, nogle gange sang de for højt. Det var en meget svingende præstation. Dessuden var der også alt for meget norsk.
\says{A} Men hvad så med skuespillet - sketchene, hvilke følelser blev stimuleret af deres optræden?
\says{L} Jeg (surt opstød) har det lidt dårligt.
\says{M} Lige mine ord. Det virkede ikke til, at de var uddannet på noget akademi.
\says{P} Næh, måske skulle man nærmere kalde dem for buhespillere.
\says{L} Men hende med frostbrysterne, hun var ellers meget nice - lækker sag.
\says{A} Det lyder da som om i har et meget ambivalent forhold til revyen, var der slet ikke noget, der imponerede jer?
\says{M} Joh, danserne var imponerende - imponerende dårlige. Det er den værste omgang usynkroniserede løben og hoppen på scenen, jeg nogle sinde har set. Det var som at være vidne til et standart afsnit af vild med dans.
\says{P} Ja, og så var revyen også alt for lang. Jeg skulle tisse hele tiden, og så alle de ordspil; tja, skulle det være op til mig, så skulle ordspil forbydes (kropssprog viser det modsatte). Og så kunne de sagtens have skåret det første ekstranummer.
\says{A} Var der slet ikke noge, der fangede øjet?
\says{L} Jo rekvisitterne - de var flotte.
\says{M} Ja, hvis de altså var lavet af en børnehaveklasse.
\says{P} Én børnehaveklasse!? Ha! Min nevø Oliver på tre år kunne hhave lavet et flottere stykke arbejde - og han går i en offentlig børnehave.
\says{L} Ja studsede lidt over instruktørerne - har de været fulde under hele forløbet?
\says{M} Nej, det er kun dig Leonora.
\scene{Alle griner}
\says{P} Instruktører - ha! En flok smagsdommere; det er hvad de er - føj!
\scene{Alle rejser sig og ninjaer tager deres stole.}
\says{A} Tja, måske? Men hvad kan vi så tage med os videre fra denne aften?
\says{L} Det er simpelt - revyen handlede egentlig bare om de tre d'er: Druk, Damer og Disney.
\says{M} Sandt! Revyen handler i bund og grund bare om efterfesten.
\says{A} Nårh ja, men lad os lige give sangerne en sidste chance for at imponere positivt.
\scene{Tæppe}

\end{sketch}
\end{document}
