\documentclass[a4paper,11pt]{article}

\usepackage{revy}
\usepackage[utf8]{inputenc}
\usepackage[T1]{fontenc}
\usepackage[danish]{babel}

\revyname{MatematikRevy}
\revyyear{2013}
\version{1.0}
\eta{$x$ minutter}
\status{udkast}


\title{Nørden Niels 2}
\author{Maling}

\begin{document}
\maketitle

\begin{roles}
\role{X}[KØK] Instruktør
\role{N}[Albert] Nørden Niels
\role{S}[Lilli] Socialrådgiver
\end{roles}

\begin{sketch}
\says{S} Ja, og jeg har nu igen fået besøg besøg af Nørden Niels. Du har jo tidligere været rus på naturvidenskab, men det er du kommet videre fra.

\says{N} Ja, det var en uheldig livsstil jeg fik startet der, men jeg har nu lagt det bag mig, det er der ikke mere af. Desuden har jeg heller ikke tid til det pjat, for nu er jeg jo blevet rusvejleder!

\says{S} Åh nej..

\says{N} Åh jo, det er slet ikke så slemt som det lyder. Vi er nogle friske unge studerende, der hver sommer tager ud i en hytte. Og så drikker vi øl!

\says{S} Det er jo slet ikke det der er meningen med rusvejleder-ansvarsposten.

\says{N} Jojo, det er jo et institut-støttet projekt.

\says{S} IMF giver jo ikke penge til rusturen for at I skal rende rundt og fulde jer, det er så de nye kan få en god social start på studiet.

\says{N} …Ah, det havde jeg ikke lige set komme, god start, aha! Men vi giver jo også de nye en øl i ny og næ!

\says{S} ...Kun hvis de siger nej!

\says{N} Nå ja, vi bryder os selvfølgelig ikke om hvis de siger nej.

\says{S} Hør her; de nye vil have en lærerig uge, hvor de i rolige omgivelser kan lære deres nye studiekammerater at kende.

\says{N} NÅ! Rolige omgivelser, det giver mening.. Men nu skal du lige tænke på, at dengang jeg var rus og var på rustur, der var det med til at gøre at jeg ikke droppede ud, eller endnu værre, skiftede til fysik.

\says{S} Nej, det er altså lige omvendt.

\says{N} HVAD!? Er der nogen der skifter fra fysik til mat’matik for at komme på rustur?!

\says{S} ...Ja, det er der faktisk.

\says{N} Ja, dét kom som et chok… Men det skal da have konsekvenser! I morgen tager jeg ned på instituttet og så aflyser jeg rusturen!

\says{S} Nej, det skal du sgu ikke, rusturen er jo en gammel tradition og mange får noget godt ud af det.

\says{N} Okay, men så holder vi nok bare en lille rustur med nogle få light øl.

\says{S} Tja, måske er rusvejleder bare ikke dig.

\says{N} Så vil jeg være rus igen.

\says{S} Det kan du ikke.

\says{N} Nå, der røg den mulighed.. Men jeg er stadig en nørd!

\says{S} Ja, dét er du!

\end{sketch}
\end{document}
