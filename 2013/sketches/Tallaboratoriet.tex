\documentclass[a4paper,11pt]{article}

\usepackage{revy}
\usepackage[utf8]{inputenc}
\usepackage[T1]{fontenc}
\usepackage[danish]{babel}
\usepackage{amsmath}

\revyname{MatematikRevy}
\revyyear{2013}
\version{1.0}
\eta{$x$ minutter}
\status{udkast}


\title{Tallaboratoriet}
\author{William, Alexander, m.fl.}

\begin{document}
\maketitle

\begin{roles}
\role{X}[NB] Instruktør
\role{R}[Signe] Rus
\role{I}[Anders] Instruktor i TalLab 1
\role{S0}[Anna] Studerende
\role{S1}[Anne G] Studerende
\role{S2}[Emil] Studerende
\role{S3}[Hektor] Studerende
\role{S4}[Ane] Studerende
\role{S5}[Mette] Studerende

\end{roles}

\begin{sketch}


\scene{R,I går ind på scene, hvor der står laboratorieagtige ting. Ann G og Mette står og summerer tal på en tavle til venstre på scenen. Hector og Jenny står og bytter plads med et skilt hver til højre på scenen. Anna sidder foran på scenen og kigger under kasserne med naturlige og reele tal. I midten er en reol med ting på.}

\says{R} Hvorfor har vi egentlig et tallaboraroium?

\says{I} Når man skal teste en formodning, er der kun en måde at gøre det på. Ved EKSPERIMENTER! Man kan jo ikke bare ”bevise” tingene en gang for alle.

\says{R} Nej, det kan der måske være noget om?

\says{I} Lad mig give dig en lille rundvisning i MatLab. Her er vi ved at bevise at summen af de første $n$ tal er lig halvdelen af tallet gange dets efterfølger. \act{Peger på Ann G og Mette} Nu har vi snart tjekket for 98\% af de kendte tal. Og vi kan derfor med rimelighed konkludere at formlen gælder generelt. Vi opdager dog stadigvæk engang imellem nye tal. 

\says{Ann G og Mette} 55!

\says{R} Men er der ikke uendeligt…. \act{Bliver afbrudt}

\says{I} Se herovre. Der står der to Eksperimentielle-mat’ere og tester hvorvidt tal kommuterer.

\scene{Går hen til Jenny og Hector. Mette går frem til Anna.}

\says{R} Og hvad laver de to?

\scene{Går hen til Anna og Mette. Jenny går hen til Ann G.}

\says{I} De leder efter imaginære tal, men det skal du ikke bekymre dig om, du har så travlt med alle dine laboratorietimer at du først skal lære om det på kandidaten. Derudover er det meget omdiskuteret hvorvidt de imaginære tal overhovedet findes. Vi har ikke fundet nogen endnu.

\scene{Mette rejser sig. Emil kommer ind med en vogn, putter en papkasse på hovedet af Hector og stiller de komplekse tal på reolen.}

\says{Mette} Har I set mit differentiale? Det forsvinder konstant, nåh det gør ikke så stor en forskel.

\says{I} Prøv at kigge i værktøjskassen ved siden at ”gange med 1” og ”læg 0 til”.

\says{Mette} Vi skal forresten også have bestilt nye gangetegn fra minen på Grønland.

\says{R} Køber I gangetegn?

\says{I} \act{Til Signe} Ja, vi prøvede at fremstille dem kemisk, men de regnede kun med to decimalers nøjagtighed.

\says{R} Hvad sker der så herovre?

\scene{Går over Ann G og Jenny}

\says{I} Herovre prøver vi at kvadrere cirklen, og vi er ved at være tæt på, men det er ikke helt lykkedes endnu.

\scene{Ann G kvadrerer cirklen.}

\says{I} Her har vi Elitestudenten. En rigtig matematisk laboratierotte. Han er også kendt som ”Den lille mand i Maple”. Prøv at se her: Hvad er forholdet mellem 137 og 80?

\says{E}

Input: 137, 80

Ratio: $\cfrac{80}{137}\approx 0,5839416058$

\says{I} I hans bachelorprojekt, differentierer han funktionen $e^x$, 38 gange!

%\says{R} Ej, nu bliver det for dumt. Jeg skulle have vidst at det var for godt til at være sandt, da jeg hørte at der var en matematikafdeling på RUC.

\says{R} Årh, det er alligevel ret sejt!

\says{I} Jaah, her har vi tydeligvis fat i den lange ende. Bare se på dem fra KU, der følger i vores fodspor med deres nye kursus 'Eksperimentel Matematik' - de vil bare altid være ét skridt bag os her på DTU!

\scene{Lys ned}

\end{sketch}
\end{document}

 

 

 

 