\documentclass[a4paper,11pt]{article}

\usepackage{revy}
\usepackage[utf8]{inputenc}
\usepackage[T1]{fontenc}
\usepackage[danish]{babel}

\revyname{MatematikRevy}
\revyyear{2013}
\version{1.0}
\eta{$x$ minutter}
\status{udkast}


\title{Øvesegment 2}
\author{}

\begin{document}
\maketitle

\begin{roles}
\role{X}[KØK] Instruktør
\role{H}[Søren] Henning (Soldat)
\role{M}[Ada] Monica (Monique) (Heks)
\role{T}[Therkel] Torben (Hund)
\role{I}[Anne G] (Instruktør)
\role{S}[Anders] (Træ)
\end{roles}

\begin{sketch}
\scene{Lys op. Torben ligger i et hundehus på scenen og øver sig}

\says{T}[Med forskellig intonation] Vov.... VOooooV. Vov! Prøv med en dobbelt måske... Vovov... VOV! Der var den sgu. Lige i skabet. Sådan skal den leveres.

\scene{I kommer ind midt i Torbens replik}

\says{I} Torben...

\says{T}[Fortsætter] Vov. Voov.

\says{I} Torben...

\says{T} Ja? 

\says{I} Hvad er det du står og laver?

\says{T} Jeg kommer lige in-character. Du ved, laver de forberedende øvelser til øvesegmentet.

\says{I} Hvor længe har du været her?

\says{T} En 8 timers tid. 

\says{I} Men den er jo 10 om morgenen!

\says{T} Man skal være velforberedt. Det er skuespillerens 13'ende regel. 

\says{I} Jeg har virkelig ikke brug for at høre de andre 12...

\says{T} Der er faktisk 101. De er i den her bog jeg har, '101 tips til en skuespiller'. \act{Holder Kalkulus frem}

\says{I} Og den beholder du bare.

\scene{H, og M kommer ind. M har stadig prinsessekjole på men et stort sort slør om hovedet.}

\says{I} Godt så er vi her alle sammen. Jeg kan se at alle har tilnærmelsesvis rigtige kostumer på idag. Så nu prøver vi, for første gang at køre det hele igennem uden papir. Vi tager scene 1, hvor soldaten møder heksen. 

\scene{T går ud.}

\says{I} Der kom en soldat marcherende hen ad landevej. Så mødte han en gammel heks på landevejen; hun var så ækel, hendes underlæbe hang hende lige ned på brystet

\says{M}[Fornærmet] Jeg er ikke ækel!

\says{H}[Nervøst] Nej, det vil jeg nu også sige. Det er ret groft sagt.

\says{I} Det er bare en rolle du spiller. Lad os nu prøve at fortsætte en gang. 

\says{M}[Ugideligt] Okay... \act{til H} Hey du... Soldat. Kan du se det der paptræ? \act{Peger på træet}  

\says{H}[Nervøst] Ja. Og jeg skal puste og pruste til det vælter.
\scene{I tager sig til hovedet.}
\says{M}[Bestemt] Nej. Kravl ned i det. Find en hund, tag dens lighter og kom tilbage op okay? Så skal du få en masse penge. 

\says{H} Jamen det lyder da dejligt.

\says{M} Så kravl ned i træet. \act{Fornærmet mod I} Så må vi se om jeg holder mit løfte. Folk lover jo en masse en ting som de ikke holder.

\scene{M går ud}

\says{I}[Mumler] Det var trods alt bedre end sidst. \act{Tydeligt} Okay, næste scene. Torben.

\scene{T kommer springende ind på scenen på alle fire}

\says{T}[Entusiastisk] Vov vovovovovov \act{Springer rundt på scenen på alle fire. Bider sig fast i H's ben.} Grrrrr

\says{H}[Skræmt] Aaaarrg \act{Forsøger at ryste H væk.} 

\says{I}[Vredt] TORBEN! Hvad i helvede har du gang i?!
 
\says{T} \act{Slipper H og rejser sig op} Jeg er 'in character'. Jeg spiller hund. 

\scene{I tager sig til hovedet i frustration.}

\says{H}[Imponeret] Nøj. Du virkede helt ægte.

\says{T} Tak skal du have, hemmeligheden ligger i savlet. Man skal huske at \act{bliver afbrudt}

\says{I}[Afbryder, mærkbart irriteret] Torben, nu siger jeg det her en sidste gang. Det eneste du skal gøre er at sidde på scenen og sige 'vov'. 

\says{T} Jamen jeg forsøger jo bare at give figuren lidt dybde. Jeg ved du vil kunne lide det når jeg afmærker mit teritorium ved at urinere op af Hennings ben.

\says{I} Der er ikke bruge for mere dybde, og du skal overhovedet ikke urinere op af nogen som helst. 

\says{T} Du kvæler min skuespillersjæl!

\says{I} Vær glad for det er det eneste jeg kvæler! Enten siger du 'vov', eller også er du fyret!

\scene{T går modvilligt over i venstre side af scenen og sætter sig.}

\says{T}[Modvilligt] 'vov'.

\says{H}[Nervøst] Nøøj sikken stor hund. 

\says{I}  Henning ville du måske have det bedre med at spille træ? Ej, du tager fyrtøjet, og så kører vi heksens dødsscene. Torben, vi siger tak til dig for idag. 

\says{T}[mumler irriteret] Proletar 

\scene{T går ud. M kommer på scenen.}

\says{M}[Til H, stadig ugideligt] Hvad, har du så mit fyrtøj?

\says{H} Ja, og jeg pustede og prustede på det, men lige lidt hjalp det.

\says{M} Henning, du skal bare hugge hovedet af mig.

\says{H}[Nervøst] Jeg er ikke så glad for det der med mord. Det er så drabeligt. 

\says{I} Jamen Henning, det er jo en vigtig del af historien.

\says{H} Det føles bare ikke rigtigt bare sådan at slå hende ihjel.

\says{I} Hug nu hovedet af hende så vi kan komme videre.

\says{H}[Mod M] Jeg er forfærdeligt ked af at måtte gøre det her


\says{I} Så huggede soldaten hovedet af heksen, og hun faldt om.

\scene{H lader som han hugger hovedet af M} 

\says{M}[Overdramatisk] Ååååh jeg dør. Var jeg dog bare født som prinsesse og ikke som heks. Så kunne mit liv måske have været reddet.

\says{I}[Irriteret] Og hun faldt om.

\says{M}[Overdramatisk] Men ak, således er livets lotteri.

\says{I} I sin udendelige frustration over hendes klynkeri hakkede soldaten hende i 15 stykker der alle sammen var ude af stand til at kommunikere.

\scene{M kigger vredt på I.}

\says{H} \act{Imens han hakker M i småstykker} Undskyld, undskyld, undskyld.

\says{I} Den næste scene er så med prinsessen. Desiré kommer du på scenen?

\scene{M rejser sig og begynder at gå ud}

\says{H} Desiré er ikke kommet idag. Det var noget med nymåne.

\says{I} Jamen, hvad i alverden skal vi så.... \act{Kalder langsomt} Moniqué?

\says{M}[Retter I]\act{Stoppe op imens hun retter I} Monica \act{Ser meget forvirret ud}

\says{I}[Forvirret] Øhm, Monica så. \act{Sødt} Hvad siger du til at tage prinsesserollen?

\says{M} Tak. Langt om længe kan jeg indtage min rette plads som prinsesse. 

\says{I} Det virker ikke rigtig. Vi stopper her for idag. 

\says{M}[Vredt] Nu kan det være nok! Jeg finder mig ikke i det. Jeg har læst 5 år på skuespilsakademiet. Jeg er en star, og du forstår dig ikke på kunst! \act{Forlader scenen}
 
\says{I} Jamen Monica...

\says{M}[Råbende fra backstage] Moniqué!

\says{I} Moniqué vent lige engang.\act{Går ud fra scene.} 

\scene{H står og kigger lidt rundt i et stykke tid.}

\says{H}[Spørgende] Er det så nu jeg skal puste og pruste?

\scene{Lys ned}
\end{sketch}

\end{document}