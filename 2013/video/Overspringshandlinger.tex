\documentclass[a4paper,11pt]{article}

\usepackage{revy}
\usepackage[utf8]{inputenc}
\usepackage[T1]{fontenc}
\usepackage[danish]{babel}

\revyname{Matematikrevy}
\revyyear{2013}
% HUSK AT OPDATERE VERSIONSNUMMER
\version{1.0}
\eta{$n$ minutter}
\status{Udkast}

\title{Overspringshandlinger}
\author{Maling}
\melody{Rasmus Rap Sluttema}

\begin{document}
\maketitle

\begin{roles}
\role{X}[KØK] Instruktør
\role{S}[Christine] Studerende
\role{R}[Julemanden] Roommate
\end{roles}

\begin{song}
\scene{S og R ligger på scenen og sover. Et vækkeur ringer, R rører sig ikke, S springer op af sengen og begynder at synge. R vågner stille og roligt op, tydeligvis irriteret over S er så energisk så tidligt. Hver eneste gang han finder en overspringshandling i sangen må der godt være en pause}

\sings{S} I dag vil jeg studere, studere, studere 
I dag vil jeg studere \act{lugter til sig selv} - Jeg tager lig’ et bad! 

\sings{S} I dag vil jeg studere, studere, studere 
I dag vil jeg studere, \act{maven rumler} - Først efter morgenmad!

\sings{S} I dag vil jeg studere, studere, studere. 
I dag vil jeg studere \act{finder en støvkost} - Men først skal gøres rent!

\sings{S}  I dag vil jeg studere, studere, studere 
I dag vil jeg studere \act{Finder en kiks} - En pause er fortjent!

\scene{S sætter sig ned, R er nu vågen, sidder stadig på gulvet og kigger ud på publikum}

\sings{R} I går sku’ han studere, studere, studere 
Men han fik intet lavet, han er jo blot en dreng

\sings{S} I dag vil jeg studere, studere, studere 
I dag vil jeg studere \act{gaber} - Men først skal jeg i seng!

\scene{De lægger sig begge to igen, bandet spiller temaet}
\end{song}
\end{document}