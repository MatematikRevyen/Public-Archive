\documentclass[a4paper,11pt]{article}

\usepackage{revy}
\usepackage[utf8]{inputenc}
\usepackage[T1]{fontenc}
\usepackage[danish]{babel}


\revyname{MatematikRevyen}
\revyyear{2016}
% HUSK AT OPDATERE VERSIONSNUMMER
\version{1.0}
\eta{$3$ minutter $45$ sekunder}
\status{Færdig}

\title{Eksamenssketch}
\author{Marie '14, Ulrik '14, Freja '15 \& Nicholas '15}

\begin{document}
\maketitle

\begin{roles}
\role{X}[Alexander] Instruktør
\role{E} Speaker
\role{S1} Studerende 1
\role{S2} Studerende 2
\role{S3} Studerende 3:
\role{SE} Søren Eilers (alternativt nogen, der spiller censor)
\end{roles}

\begin{props}
\prop{X borde}
\prop{X stole}
\prop{3 mobiltelefoner}
\prop{En stak papir afleveringer}
\prop{Pejs}
\end{props}


\begin{sketch}

\scene{Sort scene. Studerende er til eksamen, sidder ved borde.}

\says{E} Godaften og velkommen til eksaminationslokalet. I dag skal I eksamineres i et eller andet ligegyldigt kursus, som jeg aldeles ikke har forstand på. 
% Hvis i skulle gå i panik, så har vi nødudgange  hér, hér, dér og dér (evt. mørkt på scenen). Desuden fører brug af åben ild til direkte bortvisning fra eksamen. 
Hvis du har spørgsmål, så ræk blot hånden op, og en yderst behjælpelig pensionist vil komme dig til undsætning. Hvis I skal på toilettet, bedes I række hånden op. Ellers tillades det ikke, at I forlader jeres pladser. 
%Eksamen er delt op i træ dele, som kræver stigende grad af hjælpemidler. Dette inkluderer GT, Grøn Tuborg og ikke mindst øl. 
I har nu adgang til eksamen. Held og lykke.

\scene{Lys op. E træder ind på scenen for at fortælle om den nye eksamensform} 

\says{E} Vi har efter mange klager over ITX-eksamen revolutioneret jeres eksamen! Til det annoncerer jeg nu jeres første eksamensmulighed…. TINDER! Censor sidder med online, og er klar til at swipe.. jeg mener bedømme jeres opgaver løbende.

\scene{Søren Eilers kunne muligvis blive filmet og så sidde og swipe..Profilbilleder er naturligvis tilgængelige.}

\says{E} Bare find jeres telefoner frem og gå i gang!

\says{S1} OMG jeg har fået et match!!
\says{S2} \act{Rækker hånden op og taler til E} Kan du fortælle mig præcis, hvordan jeg afleverer den her opgave?

\says{E} Umiddelbart er der faktisk ikke lige nogen ligefrem, oplagt måde at aflevere en eksamensopgave via en datingApp på, det kan jeg godt se, nu du spørger… men vi har FLERE gode ideer!

\scene{E afbryder eksamen igen.}
\says{E} Min damer og herre - SPC - eksamen. - SNAPCHAT - 
\scene{Eleverne er seriøst i tvivl om, hvad fanden de skal gøre og er temmelig desperate. S1 tager selfie, S2 prøver at indspille en besvarelse af opgaven. S3 prøver at optage sin skrevne besvarelse.}
\says{S3} Ud fra hvilke kriterier bliver de her opgaver egentlig bedømt?
\says{E} Jo, ser du, jeg er da helt sikker på, at professor Søren Eilers bedømmer jeres opgaver ud fra de helt rette kriterier.
\says{SE}  Altså... ud fra de 8 sekunders video er du er dumpet… det er da virkelig en grim trøje det der.

\says{S2} Men jeg vil virkelig gerne kunne skrive! Bare et eller andet!

\says{E} Er det virkelig nødvendigt at kunne skriver noget ned? Jamen, så har jeg lige det I leder efter! Mine damer og herrer - TWIX - TWITTER eksamen! Eksamen hvor I kan formulere jer på hele 140 tegn!
\says{S1} Hvilket hashtag skal vi så bruge?
\says{E} I skal selvfølgelig hashtagge både, Københavns Universitet, kursets fulde navn og jeres eksamensnummer.
\says{S3} Hvor mange tegn tæller en ligning for?

\says{SE} Jeg kan ikke bestå ham her… det her tweet kan jo ikke engang complies i TeX.

\says{S1} Øh… hashtag Eksamen2016 er blevet trending, og jeg er endt i en diskussion med en fyrre-årig amerikaner.
\says{S2} Så lad dog være med at svare ham.
\says{S1} Nej, så vinder han.
\says{E} Der skal være stille i eksaminationslokalet!
\says{S3} Men man kan jo ikke nå at skrive noget som helst på 140 tegn!
\says{E} Det er nu taget til revision, og det er blevet besluttet, at I kan få lov at skrive alt det i vil på Arto til re-eksamen.
\says{S2} Men det findes jo ikke mere!
\says{S1} Undskyld, kan vi ikke bare få lov til at aflevere i hånden…?
\says{E} Så… I utaknemmelige skarn. Her giver vi jer den ene nyskabende eksamensform efter den anden, og intet får vi andet end brok. Hvis det skal være sådan, kan I fandeme få papireksamen tilbage på jeres lortefag.
\scene{Studerende jubler.}
\says{SE} Alt er tilbage ved det gamle - endelig kan jeg få varmen igen. \act{tænder op i pejsen, med de afleverede opgaver...}

\scene{Lys ned}

\end{sketch}
\end{document}
