\documentclass[a4paper,11pt]{article}

\usepackage{revy}
\usepackage[utf8]{inputenc}
\usepackage[T1]{fontenc}
\usepackage[danish]{babel}

\revyname{MatematikRevy}
\revyyear{2016}
\version{1.0}
\eta{$5$ minutter}
\status{Færdig}


\title{Mussoline}
\author{Freja '15, Ulrik '14, Marie '15 \& Nicolas '15}

\begin{document}
\maketitle

\begin{roles}
\role{X}[Freja] Instruktør
\role{M}[Tomas] Mussoline i en sortprikket kjole 
\role{A}[Mikkel M] Adolf med overskæg, taler med hæs stemme
\role{R}[Christoffer] Forvirret første års studerende - russer
\role{E}[Amalie T] Professor Evigglad
\role{R1}[Nicolas] Kopi af rus

\end{roles}

\begin{props}
\prop{Tændstiksæske i menneskestørrelse}
\prop{Vodkaflaske}
\end{props}

\begin{sketch}


\scene{Lys op. M ligger midt på scenen og sover i en Tændstikæske. Adolf kommer ind på scenen og vækker M.}

\says{A} Mussoline, Mussoline - Der kommer russer.
\says{M} Men det er jo ikke August endnu, Adolf.
\says{A} Ja. Vi er besetzt på alle seiten.
\says{M} Mama mia. Hvad skal vi dog stille op? De kommer her med al’ deres kop-spaghetti trods mine mange tiltag for instant rice.
\says{A} Æh, det er altså ikke alt for godt, Mussoline. Du musst komme og helfen.
Mussoline træder ud af sin tændstiksæske og gør Mussolinehonnør.
\says{M} Det må jeg vel. Hvor holder de russer egentlig til?
\says{A} Altså… jeg har dem ik’ gesehen.
\says{M} Så må vi hellere tage at finde dem.

\says{A+M}[Synger mens de går rundt] Bim bam pater, jeg er en grum diktator
Det eneste jeg drømmer om
Det er at undertrykke folk
Og afskaf’ det spaghetti-fis
Og fylde hylderne med ris
Bim bam pater, jeg er en grum diktator.
Bim bam pater, jeg er en grum diktator
 
\scene{A og M går ud af en scenekant mens de synger. Russer kommer ind fra anden scenekant med tom flaske vodka og lægger sig i Mussolines “seng”. Mussoline og Adolf kommer ind igen fra samme side, mens de stadig synger.}

\says{M} Ih altså, Adolf. Du siger nogle fjollede ting engang imellem. Først sagde du “heil,” og ikke “hej”, og nu siger du, at der er russer, men der er ikke så meget som én lærebog i syne.
\says{A} Aber Mussoline. Jeg har altså ein gesehen.  Hvor kommer de overhovedet von?
\says{M} Det er sådan et sted, der hedder gymnasiet, Adolf.
\says{A} Ih altså… er der spændende?
\says{M} Det er der sikkert, men det er ikke et sted for en ambitiøs ung mand med et usundt forhold til mål og middel. Men hvem er nu det, der ligger i min seng? 

\scene{R snorker. Mussoline rusker i æsken.}

\says{M} Hvad tror du egentlig, du laver? 
\says{R} Det er jo September, så er det tid til at drikke sig æske stiv
\says{A} Hvorfor skal vi hele tiden lave ordspil, bare fordi det er Matematikrevy?
\says{M} Er du sådan en… rus?
\says{R} Da.
\says{M} Og har du… dine kammerater med?
\says{R} Njet.
\says{M} Så du er… alene?
\says{R} Da.
\says{A} Ingen kampvogne?
\says{R} Njet.
\says{M} Ingen bomber?
\says{R} Njet.
\says{A} Ikke engang ein klein kasteskyts?
\says{R} (Holder flaske op) Molotov.
\says{A} Dann er han måske slet ikke så gefährlich?
\says{M} Så kan han måske komme med?
\says{A} Das weiss jeg nu ikke. Hvad hvis han nu er gefährlich, Mussoline?
\says{M} Pjat med dig.
\says{R} Hvad I gør?
\says{M} Vi skal ud og hverve folk til vores milits.
\says{A}[Fornøjet og børneserieagtig] Ja, og så skal vi verdensherredømmet übertage.
\says{M} Vil du være vores første mand?
\says{R} Hvad I give?
\says{M} National stolthed og fællesskabsfølelse!
\says{R} Vodka?
\says{A} Nein, nein… det er jo en børnerevy.
\says{R} \act{Bliver trist}
\says{M} Så, så. Måske, hvis det går rigtig godt kan vi også lave en revy for voksne.
\says{R} \act{Lyser op og siger optimistisk} Så vodka?
\says{M} Lige så meget, du kan sluge.
\says{R} Så jeg følge med.

\scene{A, M og R vandrer derud af i samlet flok og synger bim bam pater. De går dog ikke alt for længe, før de støder på Professor Evigglad, der sidder med en appelsin i hånden.}

\says{A} HEIL!
\says{E} \act{Taber to appelsiner af overraskelse. Han samler dem op og smiler.}Nej, den var hel. Nu er den to hele.
\says{M} Hvordan gjorde du det?
\says{E} Gjorde hvad?
\says{M} Du lavede to appelsiner.
\says{A} Ja, det var spændende!
\says{R} Da.
\says{E} Øh… det handler om at vælge noget, der er godt.
\says{M} Bare noget, der er godt?
\says{E} Ja altså… man skal faktisk være lidt kløgtig. Men hvis man drejer (evt. “drejer og deler”) smart \act{Går over og drejer på R}, så kan man løse tilsyneladende umulige problemer.
\scene{Der dukker en R mere op på scenen (rollen R1).}

\says{A} Das er jo ein endelig løsning… på das militsproblem.
\says{M} Det er nemlig rigtigt, Adolf. Vil du slutte dig til os, hr...? (evt. “og sammen med os dividere et vincere”). 
\says{E} Evigglad. De studerende elsker mig. Slutte mig til hvad, forresten?
\says{A} Et tusindårsrige! Det bliver så sjovt!
\says{E} Hvad skulle jeg med noget, der kun er endeligt?
\scene{A forsøger at svarer men M skærer ham af med ordene}
\says{M} Et millionårsrige!
\says{E} Nej, I hører jo overhovedet ikke efter.
\says{A} Et milia…
\says{M}[Afbryder A] Et uendelig-årsrige!
\says{E} Findes sådan et?
\says{M} Ikke endnu, men med din hjælp, tror jeg, at vi kan nå at erobre hele verden!

\scene{De går af scenen og synger bim bam pater}

\says{Alle} Bim bam pater, jeg er en grum diktator
Det eneste jeg drømmer om
Det er at undertrykke folk
Og afskaf’ det spaghetti-fis
Og fylde hylderne med ris
Bim bam pater, jeg er en grum diktator.
Bim bam pater, jeg er en grum diktator




\scene{Lys ned.}

\end{sketch}
\end{document}


