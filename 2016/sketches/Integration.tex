\documentclass[a4paper,11pt]{article}

\usepackage{revy}
\usepackage[utf8]{inputenc}
\usepackage[T1]{fontenc}
\usepackage[danish]{babel}

\revyname{MatematikRevy}
\revyyear{2016}
\version{1.0}
\eta{$3$ minutter}
\status{Færdig}


\title{Integration}
\author{Erik '15}

\begin{document}
\maketitle

\begin{roles}
\role{X}[Jakob] Instruktør
\role{I}[] TV-interviewer
\role{P}[] Politiker

\end{roles}

\begin{props}
\prop{2 Stole}
\prop{Cue-cards}
\end{props}

\begin{sketch}


\scene{I og P sidder over for hinanden forrest på scenen, midt-for i hver deres stoler. Evt. spiller en TV-jingle.}

\says{I} Velkommen til programmet.
\says{P} Tak.
\says{I} Og tillykke med valgsejren.
\says{P} Jo tak.
\says{I} Du er jo integrationsordfører for Ligefrem, og jeg vil derfor spørge dig: Hvad vil den nye regering gøre anderledes på udlændingeområdet?
\says{P} Enhver kan se, at det den tidligere regering har gjort ikke har virket. Indtil nu har strategien været at få sat de nye danskere i forbindelse med nogle ressourcestærke borgere i vores samfund, der kunne give dem et job. Som du kan høre handler det dybest set om at få dem med mange penge til at tage sig af problemet. Derfor kaldes denne metode også af spøgefulde sjæle for Rigmandsintegration. \\
\noindent Men hvis vi kigger rundt på landene omkring os, så er der masser af inspiration at hente. Og her tænker jeg ikke på Syldavien, hvor man blot substituerer de indsatte i fængslerne med asylansøgere. Nej, her tænker jeg på Frankrig, hvor man har haft stor succes med en tilgang, der fokuserer mere på udslusning, på at få folk videre og væk. Down the stream, så at sige. Videre op ad åen. Som de kalder det dernede: LeBæk-integration.
\says{I} Handler det her ikke bare om, at I har et andet mål end den tidligere regering? Hvorfor er jeres model bedst for samfundet generelt?
\says{P} Mål og mål. Jeg mener, at det skal stå enhver frit for at vælge det mål, der giver den bedste integration.  Men man må også huske at der er grænser for integration. Man kan ikke bare integrere hvad som helst, jeg mener, hvem som helst. Derfor tror jeg også, at det er vigtigt, at vi koncentrerer os om hvad der reelt kan lade sig gøre. Jeg mener vi har jo ikke ressourcer til alle muligheder rødder og banditter.
\says{I} I er blevet kritiseret for jeres radikaliseringspolitik. Så sent som i sidste uge var Styrbords ordfører på området ude og kalde jer ”uambitiøse” og ”slappe”. Hvad har du at sige til hans kritik?
\says{P} Overalt er der folk, der analyserer vores rationaler tæt, men de kan jo ikke separere virkelighed og fantasi. De her problemer løser jo ikke bare sig selv, om man så gav nogen en million dollars for det. Det som folk ofte glemmer, når de råber op om manglende håndfasthed, er, at man altid først må undersøge konvergensen af det man gør. Ellers kan følgerne blive fatale.
\says{I} Anerkender du, at der er folk i vores samfund, der – om ikke fuldstændigt så i hvert fald partielt – er integreret i grupper, der har ekstremale værdisæt?
\says{P} Det anerkender jeg fuldt ud, men jeg anerkender også at vores politi gør en formidabel indsats for at minimere deres handlemuligheder, selvom bibetingelserne er svære.
\says{I} Men hvem er disse personer? Hvor stammer de fra?
\says{P} Funktionsundersøgelser har vist, at det kan være meget svært at finde ud af. Faktisk er der tilfælde, hvor det slet ikke er muligt at bestemme en terrormistænkts stamtræ med de metoder, vi har lært i gymnasiet. Derfor tror jeg at billedet er mere komplekst end som så. Men som jeg altid siger: De kan bare Komme An!
\says{I} Tak. Det blev det sidste ord for i aften. Til jer seere: Vi ses i næste uge. Samme tid, et andet sted.
\scene{Evt. TV-jingle. Lys ned.}

\end{sketch}
\end{document}


