\documentclass[a4paper,11pt]{article}

\usepackage{revy}
\usepackage[utf8]{inputenc}
\usepackage[T1]{fontenc}
\usepackage[danish]{babel}


\revyname{MatematikRevyen}
\revyyear{2016}
% HUSK AT OPDATERE VERSIONSNUMMER
\version{1}
\eta{40 sekunder}
\status{Færdig}

\title{Den fordomsfrie matematiker 2}
\author{Ulrik '14 \& Erik 14}

\begin{document}
\maketitle

\begin{roles}
\role{X}[Jakob] Instruktør
\role{M} Den fordomsfrie matematiker
\end{roles}




\begin{sketch}

\scene{Lys op. M er på scenen}

\says{M} Ja, så er det mig igen. Det er som sagt mig der skal fortælle jer lidt om hvordan vi kan behandle hinanden bedre. Bare lidt. Det handler i bund og grund om at vi går så meget op de små forskelle der er på folk istedet for bare at se dem som et stort hele. Tag nu f.eks. politik. Der bliver talt så meget om farver og partier. Altså jeg ser hverken en rød eller blå politiker. Jeg ser bare en løgner. (grin) Jeg ser hverken en socialdemokrat eller konservativ. Jeg ser bare en DF’er.


\scene{Lys ned}

\end{sketch}
\end{document}
