\documentclass[a4paper,11pt]{article}

\usepackage{revy}
\usepackage[utf8]{inputenc}
\usepackage[T1]{fontenc}
\usepackage[danish]{babel}

\revyname{MatematikRevy}
\revyyear{2016}
\version{1.0}
\eta{$1$ minut}
\status{Færdig}


\title{Rus og Matematikfabrikken $1$}
\author{Ulrik '14}

\begin{document}
\maketitle

\begin{roles}
\role{X}[Freja] Instruktør
\role{M}[Amalie T] Mentor
\role{R}[Mikkel F] Rus (alle replikker som Chris fra Chris og Chokoladefabrikken)
\end{roles}

\begin{props}
\prop{2 Telefoner}
\end{props}

\begin{sketch}


\scene{Lys op. M og R står i hver deres side af scenen og taler i telefon med hinanden}

\says{M} H.C. Ørsted Instituttet, det er Mentor.
\says{R} Hej Mentor, det er rus. Jeg kan desværre ikke komme til mentormøde i dag.
\says{M} Nå, hvorfor ikke?
\says{R} Nå, men ser du mentor, jeg sku' lave den her lynopgave, og der skulle vi se på sådan nogle følger, mentor. Og det var med noget at kigge på større og større og N, mentor, faktisk skulle den gå helt ud imod uendelig, mentor, og jeg er altså ikke helt færdig endnu.
\says{M} Nu skal du altså heller ikke tjekke samtlige N, rus.
\says{R} Nej, mentor, men så er der også noget, mentor.
\says{M} Ja?
\says{R} Der var det dér DisRus, mentor, og så skulle jeg vise, at $\sqrt{2}$ er irrational, og jeg har altså ikke fået skrevet alle decimalerne ned endnu.”
\says{M} Men du kunne jo også bare føre et modstridsbevis, ik' Rus?
\says{R} Jo, men mentor, HCØ er blevet væk. Det er blevet et stort hul i jorden, mentor.
\says{M} Sludder og vrøvl, jeg sidder på HCØ lige nu. Hvis du ikke er her om $10$ minutter, så er du dumpet.
\says{R} Jeg kommer om $10$ minutter, mentor.



\scene{Lys ned.}

\end{sketch}
\end{document}


