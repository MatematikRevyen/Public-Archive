\documentclass[a4paper,11pt]{article}

\usepackage{revy}
\usepackage[utf8]{inputenc}
\usepackage[T1]{fontenc}
\usepackage[danish]{babel}

\revyname{MatematikRevy}
\revyyear{2016}
\version{1.0}
\eta{$9$ minutter $30$ sekunder}
\status{Færdig}


\title{Datalogimesterskabet}
\author{Ulrik '14 \& Julie '15}

\begin{document}
\maketitle

\begin{roles}
\role{X}[Jakob] Instruktør
\role{K}[] Kommentator
\role{J}[] Julie (pigedatalog)
\role{xXx}[] xXxJegBollerDinMor1337xXx (pwnie og 1337 h4ck0r)
\role{D}[] Delbert Datastrøm (Gammelt røvhul, kommer lige fra mors kælder)
\role{F}[] FrækFyr68 (Ifølge hans mor, en veritable sex machine)
\role{T}[] Twilight Sparkle (Rus)
\role{Kv}[] Kvinde

\end{roles}

\begin{props}
\prop{En Tavle}
\prop{Cola}
\prop{Computer el.lign. til at sende dispensationsansøgninger med}
\prop{Mur af tændstiksæsker}
\end{props}

\begin{sketch}


%Datalogdiscipliner:
%Mur af tændstiksæsker v
%Formå at komme forbi en kvinde (vel at mærke UDEN at interagere med hende) v
%Skal fejle i et induktionsbevis v
%Tegn et træ v
%Forfejl et induktionsbevis v
%Få dispensation til et fjerde eksamensforsøg i Algoritmer og Datastrukturer
%Synge i morgen er verden vor v
%Coladrikning om kap.



\scene{Lys op. I højre side af scenen sidder $K$ bag et kommentatorbord, eller står ved en pult. I venstre side af scenen står en tavle og midt på scenen er en ca. 5 cm høj mur af tændstiksæsker.}


\says{K} Godmorgen mine damer, herre og apachehelikoptere og velkommen til endnu en hæsblæsende kulmination på et helt års datalogi. Det er jo i dag, at vi i matematikrevyen skal overvære dette års Datalogmesterskab LIVE fra Store UP1, hvor vejret er fornemt, scenekanten jævn til ujævn og lyset skruet ned til datalogvenlige niveauer.
...Og  jeg hører netop i min øresnegl, at deltagerne er ved at være klar.


\scene{xXx træder ind på scenen.}


\says{K} Og første mand i manegen er pwnie-legenden xXxJegBollerDinMor1337xXx \act{udtales 'trippel x jeg boller din mor tretten syvogtredive trippel x}, der med en kampvægt på lige over 200 kg plus cola har domineret supersværvægtskategorien hele året. I alle interviews har han udtrykt både over- og undermod, og hvis vindforholdende ikke er for voldsomme, skulle han nok kunne hive en sikker sejr i land.


\scene{F træder ind på scenen.}


\says{K} Vores næste deltager er kendt fra Arto som Frækfyr68, og han er jf. sin mor en veritable sex machine, som både er fuldt funktionel og anatomisk korrekt. Han udtaler selv, at han i tæt løb med sin bror, lige akkurat missede Frækfyr69.


\scene{T træder ind på scenen.}


\says{K} Næste levende billede går under navnet Twilight Sparkle, og jeg behøver vist ikke påpege, at vedkommende er rus. Han er oppe imod et hårdt felt i dag, men studiet har ikke knækket hans kamplyst, og hvem siger i øvrigt, at man bliver en bedre datalog af at læse på DIKU?


\scene{J træder ind på scenen og retter på falsk overskæg.}


\says{K} Yes, og så har vi Julie. Julie er ikke et datalognavn. Næste?


\scene{Delbert vader sådan lidt klamt ind på scenen.}


\says{K} Og sidst men ikke mindst har vi det residerende gamle røvhul, Delbert Datastrøm, der her efter sit niende forsøg på at vinde titlen måske burde overveje at foretage sig noget andet med sit liv.


\scene{Datalogerne står lidt og kigger rundt på hinanden. De vinker lidt fjollet. Skærmer sig evt mod lyset.}


\says{K} Og vi er klar til start. Foran os venter et væld af discipliner, hvor kun den sejeste datalog kan komme igennem. Modsat DIKUs siddende studienævn, tror vi nemlig ikke på gratis point. Det er bare om at udholde de sidste pinefulde sekunder, inden signalet går...


\scene{Lyd af startsignal der går.}


\says{K} Og Julie er UDE, da hun er en pige


\scene{J er dybt utilfreds og river det falske overskæg af. Hun stormer demonstrativt af scenen.}


\says{K} Og xXxJegBollerDinMor1337xXx er stadig forpustet efter at have trådt op på scenen og har derfor lidt svært ved at komme fra start. Delbert ligger godt ben i men er desværre ikke lige så frisk, som han har været. Twilight Sparkle og FrækFyr68 ligger dog nøjagtig skulder ved skulder, idet de kommer til dagens første udfordring: Tændstiksmuren.


\scene{FrækFyr68 hopper fint over, men Twilight Sparkle snubler ind over tændstiksæskerne.}


\says{K} Og det bliver et nydeligt spring fra FrækFyr68, men Twilight Sparkle snubler lige ind i muren og bliver straks overhalet af Delbert Datastrøm med et flot skridt hen over muren. Twilight Sparkle ligger dog i vejen for xXxJegBollerDinMor1337xXx, der slet ikke kan overskue, hvad han skal gøre ved denne udfordring.


\scene{FrækFyr68 styrter videre til tavlen, hvor han giver sig i kast med kridtet.}


\says{K} I mellemtiden er FrækFyr68 nået frem til tavlen, der står i matematikkens tegn, og det er nu op til FrækFyr68 at forfejle et induktionsbevis. Og nej, altså, det starter jo superfint. Han får stillet betingelserne op og tjekket tilfældet n=0. Man kan godt se, at han ved, det ikke er helt i orden. Han står og tænker. Straks bliver han indhentet af Delbert Datastrøm, der strålende af selvtillid manuelt tjekker samtlige tilfælde for n=0 op til 5. Sjældent har Store UP1 set så ringe eksempler på induktion.


\scene{Delbert rykker videre til døren til højre for scenen, hvor han giver sig til at skråle I Morgen Er Verden Vor.}


\says{K} Og mens xXxJegBollerDinMor1337xXx og Twilight Sparkle langt om længe har kæmpet sig over tændstiksmuren, er Delbert Datstrøm fornemt i gang med at synge I Morgen Er Verden Vor. Sjældent har den gjaldet så tonedøvt her i auditoriet. Det er en fryd at overvære. I mellemtiden har FrækFyr68 langt om længe fundet ud af ikke at foretage induktionsskridtet rigtigt, og med sine medkonkurrenter lige i halen skynder han sig videre for at synge. Bemærk, hvor selvsikkert Twilight Sparkle ikke den fjerneste anelse har om, hvad induktion er, og han er fluks videre. xXxJegBollerDinMor1337xXx skal dog lige hidkalde sig den dårlige idé, før han selv kan komme med .


\scene{Delbert stopper op og stirrer lamslået på det hunkønsvæsen, der står foran Texnikken.}


\says{K} Åh nej… Delbert stopper målløs op overfor den ubestridt sværeste disciplin i hele konkurrencen - han skal forbi kvinden og videre til coladrikningskonkurrence. Mangt en datalog har givet sig i kast med denne udfordring over årene, men kvindens mytiske status på datalogisk institut gør det som regel mere end umuligt at komme inden for så to meter af dem uden at falde måbende sammen.


\scene{Delbert står paralyseret, mens de andre konkurrenter får sunget færdig og styrter videre og overhaler ham. Undtagen T, som hiver sin mobil frem.}


\says{K} Og Alle deltagere ligger utroligt tæt, bortset fra Twilight Sparkle, som febrilsk prøver at google teksten til I Morgen Er Verden Vor. Hans rusvildledere må være meget skuffet. FrækFyr68 ser endda ud til at ville gøre et forsøg på at komme forbi kvindfolket.


\scene{F træder frem og giver forsigtigt hånden til Kv. Han siger noget til hende.}


\says{K} Og vi er allerede nået til aftenens anden diskvalifikation, for ingen ægte datalog ville snakke med nogen eller noget, der ikke har et Y-kromosom. FrækFyr68 må derfor nyde resten af forestillingen fra sidelinjerne.


\scene{F ser lidt slukøret ud, men får rent faktisk et smil og et kys på kinden fra pigen, og de løber smilende ud af UP1 sammen.}


\says{K} Jamen, hvad er dog det? Pigen gengælder FrækFyr68’s tilnærmelser. Hvilken utroligt rørende, sød og gennemført udatalogisk afslutning på hans deltagelse her i foretaget. Samtidigt åbner det fuldstændigt for løbet igen og Delbert lægger sig stærkt i spidsen på vej ind i cola-drikning ,skarpt efterfulgt af xXxJegBollerDinMor1337xXx, der ligger godt i svinget og får sig en kraftig første tår. Sidst og absolut mindst kommer Twilight Sparkle flyvende efter han febrilsk fik stammet sig igennem en næsten vederstyggeligt toneren udgave af I Morgen Er Verden Vor. Han burde nok have hørt lidt bedre efter inden forestillingen.


\scene{Alle kombatanter drikker lynhurtigt, men xXxJegBollerDinMor1337xXx er dog meget hurtigere end de andre.}


\says{K} Det er kraftig slubren, der foregår i auditoriet i aften, og jeg har ikke set tre ansigter så røde, siden jeg blottede mig for en flok ruspiger sidste år. Vores supersværvægtsmester er dog lige en kaliber eller to bedre end siden konkurrenter og kan ubesværet vade videre til næste udfordring. Og hvad er det?! Twilight Sparkle giver op og hælder resten af cola’en ud. Dette er selvfølgelig dybt ureglementeret og uhørt, men jeg kan ikke se det mindste, vi kan stille op overfor denne grumme usportslighed. Og stakkels Delbert Datastrøm drikker fortsat stille og roligt af sin flaske, mens hans konkurrenter er nået til aftenens næstsidste datalogiske udfordring: At sende en dispensationsansøgning om et fjerde eksamensforsøg til algoritmer og datastrukturer.


\scene{Delbert formår endelig at få gjort sin cola færdig, men er blevet træt og går derfor langsomt tilbage til scenen.}


\says{K} Mens Delbert Datastrøm er uendeligt langt bagud, oplever vi, at Twilight Sparkle overhovedet ikke kan finde den korrekte formular på KU-net. Det tegner ikke godt, mens xXxJegBollerDinMor1337xXx selvsikkert finder den rette ansøgning, men hvad er nu det? Den bliver afvist! Det viser sig nemlig, at xXxJegBollerDinMor1337xXx, modsat langt størstedelen af DIKU’s befolkning allerede har bestået algoritmer og datastrukturer. Han må derfor selvfølgelig forlade mesterskabet her på falderebet, og dysten står nu mellem et gammelt røvhul og en rus, der trods massivt press endnu ikke er droppet ud.


\scene{T jubler åbenlyst og begynder at udfylde formularer, mens D bare kommer og indtaster sin svenske nummerplade.}


\says{K} Og mens Twilight Sparkle endelig har fundet de korrekte dokumenter, stryger Delbert Datastrøm igennem dispensationsansøgningen - studienævnet kender ham jo så godt i forvejen. Og det gamle røvhul ligger dermed i spidsen hen mod målstregen, hvor én sidste udfordring venter: At tegne… et træ. Og de er i gang. De skribler og skribler, og Delbert Datastrøm er kun lidt foran - hans gamle hånd er ikke lige så hurtig på en tavle, som de begejstrede men upræcise fakter fra vores rus her. Og det ser ud til at blive tæt løb her på falderebet, og de bliver færdige præcis samtidigt! Men hvad er nu det? Twilight Sparkle har tegnet… et rigtigt træ, og vi må derfor erklære Delbert Datastrøm SEJRHERREN af datalogimesterskaberne 2016. Så kan han måske endelig komme videre med sit liv. Det var alt for i år, men husk,  at der stadig er masser af fjollede dataloger, som du kan observere i din hverdag.


\scene{Folk forlader scenen og lyset fader.}


\end{sketch}
\end{document}


