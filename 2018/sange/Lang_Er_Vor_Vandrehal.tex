\documentclass[a4paper,11pt]{article}

\usepackage{revy}
\usepackage[utf8]{inputenc}
\usepackage[T1]{fontenc}
\usepackage[danish]{babel}

\revyname{Matematikrevy}
\revyyear{2018}
\version{2.0}
\eta{$3$ minutter}
\status{Færdig}

\title{Lang Er Vor Vandrehal}
\author{Ida F '16}
\melody{Brødrene Olsen -- Smuk som et stjerneskud}

\begin{document}
\maketitle

\begin{roles}
\role{X}[Patrick] Instruktør
\role{Y}[Kaspian] Koreograf / Danser
\role{S1}[Marie] Sanger
\role{S2}[Rikke] Sanger
\role{D2}[Emil] Danser
\role{D3}[Line] Danser
\role{D4}[Maja] Danser
\role{D5}[Nina] Danser
\role{D6}[Philip] Danser
\end{roles}

\begin{song}
\sings{S2} I den grønne park, står og kigger op
HCØ jeg ser dig
Op ad trappen går, foran døren står, ja
Svinger studiekortet

Drejer ind til højre og går ned
Ad den gang, der bar' blir' ved og ved

\sings{S1+S2} Lang er den vandrehal
Kold i et efterår
Smelteovn i juni
År efter år

\sings{S1} Gennem gitteret ser, jeg kantinen der, ja
Lækre wienerbasser
Bag mosaikken bor, i orange flor
Mit yndlingsauditorium

Solens stråler strømmer lige ned
På den gang, der bar' blir' ved og ved

\sings{S1+S2} Lang er den vandrehal
Glas og beton
Længere ser den ud
År efter år

En vandrehal
Men'sker jeg finde kan
Søndag ved midnatstid
Og dag efter dag

\sings{S1} Smuk er den vandrehal
\sings{S2} Giver komfort
\sings{S1} Li'som et andet hjem
\sings{S2} År efter år

\sings{S1+S2} Smuk er vor vandrehal
Som tiden går
Smukkere ser den ud
År efter år

\sings{Kun lørdag} Tag til vor vandrehal
Ikk' indre byyy
Kom med til efterfest 
Efter revy

Til efterfeeest
I vandrehallen efter matrevyyy

\sings{dette i stedet for de sidste 2 linjer om fredagen} Vor vandrehaaal
år efter år efter år efter år efter år
\end{song}

\end{document}
