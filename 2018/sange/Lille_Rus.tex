\documentclass[a4paper,11pt]{article}

\usepackage{revy}
\usepackage[utf8]{inputenc}
\usepackage[T1]{fontenc}
\usepackage[danish]{babel}

\revyname{Matematikrevy}
\revyyear{2018}
\version{2.0}
\eta{$2.5$ minutter}
\status{Færdig}

\title{Lille Rus}
\author{Anne G '11, Patrick '13}
\melody{Stig Rossen -- Menneskesøn}

\begin{document}
\maketitle

\begin{roles}
\role{X}[Anne] Instruktør
\role{S1}[KE] Sanger
\role{S2}[Stine T] Sanger
\role{R}[Rikke] Rus
\role{P}[Marie] Statist
\role{P}[Arnvig] Statist
\role{P}[Daniel] Statist
\role{P}[Eigil] Statist
\role{P}[Nanna] Statist
\end{roles}

\begin{song}
\sings{S1} Åh vær rede på at lære
For her på matematik
Har vi bare de aller bedste fag
\sings{S2} Du er startet på et studie
Mange andre falder fra
Men med flittighed og vilje
En sciencegrad du ta'r

\sings{S1+S2} Lille rus Velkommen her
Lad din viden sprede sig
Snart har du en masse lært
Lille rus du bli'r en kandidat

\sings{S1} Om du bliver rusvejleder
\sings{S2} Eller mentor på din vej
\sings{S1} Må du godt nok aldrig glemme
\sings{S1+S2} At matematik er sejt

\sings{S1+S2} Lille rus Velkommen her
Lad din viden sprede sig
Snart har du en masse lært
Lille rus du bli'r en kandidat

\sings{S2} Du lærer algebra
Og måske også MI
Og bliver den instruktor vi kan li'
\sings{S1} Alle drømmene du drømte
En tirsdag på Caféen?
\sings{S1+S2} Nu er tiden næsten inde
Det er aldrig for sent

\sings{S1+S2} Lille rus Velkommen her
Lad din viden sprede sig
Snart har du en masse lært
Lille rus du bli'r en kandidat

Lille rus
Lille rus er blevet kandidat
\end{song}

\end{document}
