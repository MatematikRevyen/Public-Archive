\documentclass[a4paper,11pt]{article}

\usepackage{revy}
\usepackage[utf8]{inputenc}
\usepackage[T1]{fontenc}
\usepackage[danish]{babel}

\revyname{Matematikrevy}
\revyyear{2018}
\version{1.0}
\eta{$2$ minutter}
\status{Færdig}

\title{Vi Er Alle Russer}
\author{Ulrik '14}

\begin{document}
\maketitle

\begin{roles}
\role{X}[MaWeK] Instruktør
\role{P}[Sommer] Præst
\role{K1}[KE] Kor
\role{K2}[Arnvig] Kor
\role{K3}[Stine L] Kor
\role{K4}[Sunniva] Kor
\end{roles}

\begin{props}
\prop{Rekvisit}[Person, der skaffer]
\end{props}

\begin{sketch}
\scene{Lys op.}

\scene{Der står en prædikestol på scenen. P, klædt i rober, træder op til prædikestolen og stirrer olmt på publikum.}

\says{P} Vi er alle russer.
Vi må huske på arverussen, som vi modtager idet vi træder ind i dette studieliv, som blev os påduttet, da Ahmes og Euklid i hin tid blev bortvist fra fjerdes have, til stueetagen udenfor al faglig identitet, hvor de måtte vandre i enhedscirkler i hvad, der efterfølgende af alle er blevet kendt som vandrehallen.

\says{P} Vi er alle russer.

\says{P} Og da er det ikke vores plads at dømme vore medrusser, men blot vide, at vi selv skal dømmes ved både eksaminer og opgaver i kurser med løbende evaluering.

\says{P} Lad os sammen bekende vores matematiske tro:

\scene{AV viser teksten.}

\says{P} Vi forsager Einstein \\
Og alle hans pseudoberegninger \\
Og alle hans summationskonventioner \\
Vi tror på Pythagoras, \\
Den trekantede \\
Hilbertrummet og orthogonalitetens skaber \\
Vi tror på Søren Eilers, \\
Den flotteste forelæser \\
Vor danser \\
Som er undfanget ved legoklodser \\
Født af trangen til at farve grafer \\
Pint under adskillige obligatoriske kurser i datalogi \\
Korsfæstet, død og begravet \\
Flyttet til Sverige \\
På hverdage tilbage i Danmark \\
Opfaret til anden, siddende ved professorers, de alvidendes, højre, respektivt venstre hænder, alt afhængigt af respektiv kontorbeliggenhed, \\
Hvorfra han skal komme at dumpe tænkende og russer \\
Vi tror på udvalgsaksiomet \\
Den generelle, abstrakte nonsens \\
Hvad end der står på arXiv \\
Vejlederens rettelser, \\
Instruktorens tugten, \\
Og det evige flueknepperi. \\
Firkant

\scene{Lys ned.}
\end{sketch}

\end{document}
