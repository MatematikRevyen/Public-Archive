\documentclass[a4paper,11pt]{article}

\usepackage{revy}
\usepackage[utf8]{inputenc}
\usepackage[T1]{fontenc}
\usepackage[danish]{babel}

\revyname{Matematikrevy}
\revyyear{2018}
\version{3.0}
\eta{$5$ minutter}
\status{Færdig}

\title{Disneyfigurer På Matematik}
\author{Erik '15, Ulrik '14, Sofie '15, Chris '17}

\begin{document}
\maketitle

\begin{roles}
\role{X}[Anne] Instruktør
\role{B}[Emil] Bruce Banner fra Hulk
\role{D}[Maja] Dory fra Find Nemo
\role{Yo}[Anders] Yoda fra Star Wars
\role{Po}[Nanna] Pocahontas fra Pocahontas
\role{Pe}[Sunniva] Peter Pan fra Peter Pan
\role{A}[Toke] Ariel fra Den Lille Havfrue
\role{Pi}[Daniel] Pinnochio fra Pinnochio
\role{O}[KE] Gaston fra Skønheden og Udyret
\role{N}[Line] Ninja
\end{roles}

\begin{props}
\prop{To borde}
\prop{Fire stole}
\prop{To tavler}
\prop{Stor blyant}
\end{props}

\begin{sketch}
\scene{Lys op.}

\scene{Vi er i kantinen, hvor der foregår studiecafé. Bruce Banner, Yoda og Pocahontas render rundt i lektiecafé-trøjer.  Dory sidder og har finnen i vejret. Peter Pan sidder og drysser blyantsrester ud over papirfly og sender dem afsted. Ingen er specielt koncentrerede. Ariel sidder halvt på en stol og spjætter som en døende fisk.}

\scene{Bruce bevæger sig over til Dory med et stort smil, klar til at hjælpe.}

\says{Bruce} Og hvad kan jeg så hjælpe dig med?

\says{Dory} Jo, altså. Jeg ville gerne høre, hvornår der er lektiecafé?

\says{Bruce}[Forvirret] Altså... nu har du jo været i studiecaféen den sidste halve time, Dory.

\says{Dory}[Begejstret] Ej, hvor fedt. Jeg har nemlig et spørgsmål, jeg har gået og tænkt over hele ugen!

\scene{Bruce vifter hende videre med hånden.}

\says{Bruce} Som er?

\says{Dory} Ja... så... øh... ja... Hey! Kan du ikke sige mig, hvornår der er lektiecafé?!

\says{Bruce} Ja... den næste er så på fredag.

\scene{Bruce går sin vej, og Pinnochio vandrer ind på scenen med snore på og prøver at nå et bord, men kan ikke. Han slås lidt med det til snorene går i stykker, hvorefter Yoda kommer over til ham for at hjælpe.}

\says{Yoda} Mrhm... brug for hjælp, du har?

\says{Pinnochio} Altså... der er det her problem, der binder lidt.

\says{Yoda} Mrhm... lavet noget, du har?

\says{Pinnochio}[Mens hans næse vokser] Tja... jeg har prøvet...

\scene{Yoda Slår Pinnochio med sin stok.}

\says{Yoda} Gør... eller gør ikke. Der er ikke noget prøve.

\says{Pinnochio} Jamen... det er bare det her $\varepsilon$-hejs. Det er temmelig indviklet.

\scene{Yoda hiver en tavle over, hvor kontinuitetsdefinitionen står skrevet rigtigt.}

\says{Yoda}[Mens han peger på det led, han omtaler] Mrhm... men definitionen du forstår? For hvert epsilon, med $\delta$ parere det du kan.

\says{Pinnochio}[Måbende] Øh...

\scene{Pocahontas kommer over, og jager Yoda væk.}

\says{Pocahontas} Ham skal du ikke tage dig af. Ej... han gør det bare heller ikke særligt pædagogisk med den dér ene hvide farve. Nu skal du se.

\scene{Pocahontas hiver komisk mange farvekridt frem og begynder at give alle symbolerne en unik farve.}

\says{Pocahontas} Man forstår det meget bedre, hvis lige man tager alle vindens farver i brug. Synes du ikke?

\says{Pinnochio}[Mens hans næse vokser.] Øh... ja... klart, det giver god mening.

\says{Pocahontas} Og hvad kan jeg så hjælpe dig med?

\says{Peter Pan} Jo altså... det er det her kursus, ik'? Og alt gik flyvende... indtil det ikke gjorde.

\says{Pocahontas} Ja... hvad er problemet?

\says{Peter Pan} Ja... jeg nåede til opgaverne.

\says{Pocahontas} Har du prøvet at tænke på noget godt?

\says{Peter Pan}[Misfornøjet] Bare noget, der er godt?

\says{Pocahontas} Altså... sådan... det bedste, du ved?

\says{Peter Pan}[Rasende] Jamen, siden hvornår har det hjulpet på noget som helst?

\scene{Peter Pan rejser sig og stormer ud, mens Bruce når videre til Ariel. Gaston kommer stolt ind fra bagtæppet}

\says{Gaston} Hahaa. Ingen laver opgaver som Gaston!

\scene{Gaston smækker sin læsning på bordet.}

\says{Yoda} Hmm.. Ingen dividerer med $0$ som Gaston.

\scene{Gaston forlader scenen i skam}

\says{Bruce}[Stadig smilende] Ja... hvad er så dit problem?

\scene{Beat.}

\says{Bruce}[Smilet bliver lidt mere tvungent] Du behøver ikke at være genert...

\scene{Beat.}

\says{Bruce}[Arrigt] Jamen, hvad fanden er problemet?

\scene{Ariel falder død om.}

\scene{I rasseri forvandles Bruce til Hulk}

\says{Hulk}[Vredt - d'uh] Hulk smadre! Russer!

\says{Dory}[Lyser op] Hey! En hval!

\says{Dory}[På hvalsprog] Hvornår er der lektiecafé?

\scene{Hulk brøler og skal til at banke Dory, men bliver stoppet af Yoda og Pocahontas.}

\says{Hulk} Hm?

\says{Dory} Det er godt nok ikke nemt at være Disney-karakter på matematik.

\says{Pinnochio} Så skulle du se Mulan. Hun læser datalogi.

\scene{Lys ned.}
\end{sketch}

\end{document}
