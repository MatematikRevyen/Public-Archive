\documentclass[a4paper,11pt]{article}

\usepackage{revy}
\usepackage[utf8]{inputenc}
\usepackage[T1]{fontenc}
\usepackage[danish]{babel}

\revyname{Matematikrevy}
\revyyear{2018}
\version{2.1}
\eta{$3$ minutter}
\status{Færdig}

\title{Det Tredje Studieår}
\author{Ulrik '14}

\begin{document}
\maketitle

\begin{roles}
\role{X}[Patrick] Instruktør
\role{N2}[Sunniva] Andenårsstuderende
\role{N3}[Felix] Tredjeårsstuderende
\role{N5}[Toke] Femteårsstuderende
\role{N6}[Taus] Sjetteårsstuderende
\role{N}[Eigil] N-teårsstuderende
\end{roles}

\begin{props}
\prop{Rekvisit}[Person, der skaffer]
\end{props}

\begin{sketch}
\scene{Lys op.}

\scene{N2 bevæger sig ind på scenen. Han fløjter og er fornøjet.}

\says{N2} Tænk engang. En hel sommerferie væk fra det her sted. Jeg synes ellers lige, at jeg var nået sådan en dejlig fred ved bare, du ved... at sidde i kantinen og... hygge sig ude på de stille aftentimer med de kandidatstuderende.

\scene{N2 stopper som vedkommende når til scenekanten og tager sig til ørende.}

\says{N2} Av... hvor det larmer. Og... hvor er der mange førsteårsstuderende...

\scene{N3 sniger sig ind på scenen. N3 er dystert til mode.}

\says{N3} Det er skræmmende, hva?

\says{N2} Ja... jeg husker slet ikke det her fra sidste år.

\says{N3} Jeg har talt med nogle af de ældre studerende, og de siger, at russerne overtager HCØ år for år. Snart vil der slet ikke være plads til rigtige matematikere på HCØ.

\says{N2} Abahva'? Hvem har du dog hørt det fra?

\scene{Straks begynder The Imperial March at spille, og N5 og N6 kommer ind i nazist uniformer og trækker bagtæppet tilbage for N, der tydeligvis er i føreruniform.}

\says{N}[Arrigt] HCØs folk! Jeres land er blevet taget fra jer! Hvis ikke I hæver jer over disse forbryderiske udyr - disse førsteårsstuderende - så vil E-bygningens forfald fortsætte ud i det uendelige!

\says{N2} Det... det lyder altså lidt overdrevent.

\says{N5+N6} Tavshed, når den $n$'te-års studerende taler!

\says{N} Har du måske ikke bemærket, at ZFC har lidt nederlag til Matematik Økonomernes fodboldhold?

\says{N3} Jo!

\says{N5} Tydeligt russernes skyld!

\says{N2} Ej, hør... det er altså absurd.

\says{N} Der er bestemt intet absurd ved det. Russernes blotte tilstedeværelse korrumperer og forvrænger. Matematikerens sind er rent! Russerne er inficeret af den virkelige verden. Vi har brug for plads. Matematikerne behøver denkenraum!

\says{N6} Og derfor vil vi overtage |RUM|!

\says{N3} Men hvad gør vi ved problemet med russerne?

\scene{N5, N6 og N stimler sammen og hvisker lidt.}

\says{N}[rømmer sig] Vi vil tvinge alle russer til at bære tydeligt mærke.

\scene{N5 og N6 træder nærmere og stempler N2 i panden.}

\says{N2} Men... jeg er andenårsstuderende!

\says{N3+N5+N6} Én gang rus altid rus.

\scene{N2 hyler og forlader scenen.}

\says{N3} Men hvordan forhindrer vi russerne i at fylde det hele?

\says{N6} Vi kan eksportere dem alle til Madagascar!

\says{N5} Eller bare til RUC? Eller KUA?

\says{N6} Vi kan smide dem i ovnen i S01!

\says{N} Først efter kl. 15. Vi skal huske på det administrative personale.

\says{N5} Klart, klart.

\says{N3} Vi kunne... vi kunne invadere Polen!

\scene{Alle tre nazistklædte gamle røvhuller stirrer meget underligt på N3.}

\says{N} Nu er du kraftædeme useriøs.

\scene{N5, N6 og N begynder at gå ud af scenen. Lys går gradvist ned. N3 følger efter.}

\says{N3} Ej, kom nu. Jeg mente det ikke. Vi kunne også sende russerne ind og se biorevy i næste uge!

\scene{Lys ned.}
\end{sketch}

\end{document}
