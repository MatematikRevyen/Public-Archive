\documentclass[a4paper,11pt]{article}

\usepackage{revy}
\usepackage[utf8]{inputenc}
\usepackage[T1]{fontenc}
\usepackage[danish]{babel}

\revyname{Matematikrevy}
\revyyear{2018}
\version{2.0}
\eta{$0.5$ minutter}
\status{Færdig}

\title{En Velordnet Sætning}
\author{Ulrik '14}

\begin{document}
\maketitle

\begin{roles}
\role{X}[Patrick] Instruktør
\role{Sp}[Arnvig] En meget, meget træt semi-ansvarlig
\role{P1}[Khalil] Publikummer
\end{roles}

\begin{props}
\prop{Rekvisit}[Person, der skaffer]
\end{props}

\begin{sketch}
\scene{Lys op.}

\scene{En enlig person træder ind på scenen. Et teorem bliver sat op på projekter}

\says{Sp} Matematikrevyen præsenterer: En velordnet sætning.

\scene{Den enlige person begynder at gestikulere ud i luften, mens teoremet bliver lavet om til velordnet.}

\says{P} At da de der dual-rummet eksisterer element er et et et forsvinder funktioner givet Hausdorff hensyn i integration kompakt kontinuerte. Lad lad lokalt med $\mu$, $\mu$ og $\varphi$, $\varphi$ på på Radonmål rum således, til til uendelig ved ved være være $X$ $X$ $X$

\scene{Den enlige person bukker.}

\scene{Lys ned.}
\end{sketch}

\end{document}
